% dvips -t letter lab_beachball.dvi -o lab_beachball.ps ; ps2pdf lab_beachball.ps
\documentclass[11pt,titlepage,fleqn]{article}

\usepackage{amsmath}
\usepackage{amssymb}
\usepackage{latexsym}
\usepackage[round]{natbib}
%\usepackage{epsfig}
\usepackage{graphicx}
\usepackage{bm}

\usepackage{url}
\usepackage{color}
%\usepackage{hyperref}

%--------------------------------------------------------------
%       SPACING COMMANDS (Latex Companion, p. 52)
%--------------------------------------------------------------

\usepackage{setspace}    % double-space or single-space
\usepackage{xspace}

\renewcommand{\baselinestretch}{1.2}

\textwidth 460pt
\textheight 690pt
\oddsidemargin 0pt
\evensidemargin 0pt

% see Latex Companion, p. 85
\voffset     -50pt
\topmargin     0pt
\headsep      20pt
\headheight   15pt
\headheight    0pt
\footskip     30pt
\hoffset       0pt

\input{commands_letters}
\input{commands_uaf}
\input{commands_carl}

\newcommand{\blank}{xxxx}

\newcommand{\cyear}{2026}

% provide space for students to write their solutions
\newcommand{\vertgap}{\vspace{1cm}}

\graphicspath{
  {./figures/}
}

\newcommand{\cltag}{GEOS 626/426: Applied Seismology, Carl Tape}
\newcommand{\ptag}{{\bf \textcolor{magenta}{[GEOS 626]}}}
%\newcommand{\ptag}{}

\bibliographystyle{agufull08}

\newcommand{\howmuchtime}{Approximately how much time {\em outside of class and lab time} did you spend on this problem set? Feel free to suggest improvements here.}

%------------------------------------------------------

% modes
\newcommand{\tnl}[2]{\mbox{$_{#1}\ssT_{#2}$}}
\newcommand{\snl}[2]{\mbox{$_{#1}\ssS_{#2}$}}
\newcommand{\snlm}[3]{\mbox{$_{#1}\ssS_{#2}^{#3}$}}
\newcommand{\Tnl}{\mbox{${}_nT_l$}}   % eigenfun
\newcommand{\Wnl}{\mbox{${}_nW_l$}}   % eigenfun
\newcommand{\Ynl}{\mbox{${}_nY_l$}}   % spherical harmonics
% eigenfrequencies
\newcommand{\omnl}{\mbox{${}_{n\hspace{-0.2 mm}}\omega_l$}}
\newcommand{\fnl}{\mbox{${}_{n\hspace{-0.2 mm}}f_l$}}
\newcommand{\omnlm}[1]{\mbox{${}_{n\hspace{-0.2 mm}}\omega_l^{#1}$}}
\newcommand{\fnlm}[1]{\mbox{${}_{n\hspace{-0.2 mm}}f_l^{#1}$}}

\newcommand{\cutoff}[1]{{#1}_{x\bar{n}}}
\newcommand{\Wcol}{\textcolor{blue}{W}}
\newcommand{\Tcol}{\textcolor{red}{T}}

%\newcommand{\rfind}{{\tt fzero}}
\newcommand{\rfind}{{\tt brentq}}
%\newcommand{\sfind}{{\tt ode45}}
\newcommand{\sfind}{{\tt solve\_ivp}}
\newcommand{\maxstep}{{\tt max\_step}}

%------------------------------------------------------


% change the figures to ``Figure S3'', etc
\renewcommand{\thefigure}{L\arabic{figure}}
\renewcommand{\thetable}{L\arabic{table}}
\renewcommand{\theequation}{L\arabic{equation}}
\renewcommand{\thesection}{L\arabic{section}}

\renewcommand{\baselinestretch}{1.25}

% note: carlcommands is loaded in hw626_header.tex
\input{waltcommands}
\newcommand{\fvect}{\textcolor{red}{\mbT}}
\newcommand{\fvecb}{\textcolor{blue}{\mbB}}
\newcommand{\fvecp}{\textcolor{black}{\mbP}}
\newcommand{\fvecn}{\mbN}
\newcommand{\fvecs}{\mbS}

%--------------------------------------------------------------
\begin{document}
%-------------------------------------------------------------

\begin{spacing}{1.2}
\centering
{\large \bf Lab Exercise: Hands on the moment tensor beachball [beachball]} \\
\cltag\ \\
%Assigned: February 13, 2014 --- Due: February 20, 2014 \\
Last compiled: \today \\
\end{spacing}

%------------------------

\vspace{-4.4cm}
\begin{center}
\includegraphics[width=8cm]{DoubleCoupleVectorField.eps}
\end{center}

%------------------------

\vspace{-1.8cm}
\section*{Instructions}

\begin{itemize}

\item Background reading:
%
% WHAT ABOUT AKI AND RICHARDS?
\begin{itemize}
\item Section 4.2--4.4 of \citet{SteinWysession}
\item Ch.~9 of \citet{ShearerE2}
%\item \citet{TapeTape2012beach}
\end{itemize}

\item I will loan you a painted wooden model of a double couple moment tensor (``beachball''), as well as five dowel segments.

\item A symmetric matrix is guaranteed to have three real eigenvalues. We follow the convention of eigenvalue ordering $\lambda_1 \ge \lambda_2 \ge \lambda_3$ with corresponding ordered eigenvectors \fvect\ (``tension axis''), \fvecb\ (``neutral axis''), \fvecp\ (``pressure axis'') (so $\lambda_t \ge \lambda_b \ge \lambda_p$). For this lab, we consider the double couple moment tensor, for which
%
\begin{eqnarray*}
\lambda_1 &=& \lambda
\\
\lambda_2 &=& 0
\\
\lambda_3 &=& -\lambda
\end{eqnarray*}
%
where $\lambda > 0$.

\item We follow the seismological convention that quadrants containing the tension axis (\fvect) are colored solid, while the quadrants containing the pressure axis (\fvecp) are colored white.

\item A moment tensor can be written as \citep[][eq.~1]{TapeTape2012beach}
%
\begin{equation}
M = U D U^{-1}
\end{equation}
%
where $U = [\fvect\; \fvecb \; \fvecp]$ is the (orthogonal) orientation matrix and 
%
\begin{equation}
D = \begin{pmatrix} \lambda_1 & 0 & 0 \\ 0 & \lambda_2 & 0 \\ 0 & 0 & \lambda_3 \end{pmatrix}
\end{equation}
%
contains the eigenvalues that characterize the source type of the moment tensor. $M$ is the moment tensor in its standard basis; $D$ is the moment tensor in its eigenbasis.

\end{itemize}

%------------------------

%\pagebreak
\section*{Exercises}

\begin{enumerate}
\item Insert three dowels to give an eigenframe for the moment tensor. Use a red dowel for the tension vector \fvect, a blue for the neutral vector \fvecb, and a yellow for the pressure vector~\fvecp. If necessary, reverse one vector so that the triple \fvect, \fvecb, \fvecp\ is right-handed.

%---------------------------

\item Rotate the ball \dg{180} about any one of \fvect, \fvecb, or \fvecp.
%
\begin{enumerate}
\item Does the eigenframe change? (Think: eigenframe = dowels)

\item Does the moment tensor change? (Think: moment tensor = beachball)

\item How many orientations (rotation matrices) are there for each moment tensor?

\end{enumerate}

%---------------------------

\item

\begin{enumerate}
\item Insert a black dowel for the fault normal vector \fvecn. How many possibilities are there?

\item Now insert a black dowel for the slip vector \fvecs. There is only one possibility. The slip vector points in the direction of the hemisphere (or ``block'') containing the normal vector.

\item Describe the motion of the fault blocks associated with your \fvecn\ and \fvecs.

\item How does the motion change if you interchange \fvecn\ and \fvecs?

\item How does the motion change if you reverse both \fvecn\ and \fvecs?

\end{enumerate}

%---------------------------

\item Remove \fvecn\ and \fvecs\ for now. the Change the orientation of the beachball (moment tensor) in some way, and let $\omega$ be the angle, in matrix space, between the initial moment tensor and the final moment tensor. Then $\omega$ measures how different the reoriented ball is from the original ball; $\omega=0$ means the two balls are identical, $\omega=\pi$ means the balls are negatives of each other.

\begin{enumerate}
\item Describe a rotation that will give $\omega=\pi$.

\item Describe a non-trivial rotation that will give $\omega=0$.

\item Convince yourself that if the reorientation is a rotation of \dg{90} about \fvecp, then $\omega$ is small to moderate.

\item CHALLENGE: Derive $\omega=\pi/3$ for a rotation of \dg{90} about \fvecp.

\item CHALLENGE: Derive $\omega=\pi/3$ for a rotation of \dg{90} about \fvect.

\end{enumerate}

%---------------------------

%\item Rotate the ball \dg{90} about \fvecb. Find $\omega$.

%---------------------------

%\item Rotate the ball \dg{180} about \fvecn\ or \fvecs. Find $\omega$.

%---------------------------

\item Having finished this lab and also \verb+lab_mt.pdf+, where we looked at the moment tensor as a linear transformation and a vector field, let's now look at the figure at the start of this lab. Describe it as completely as possible. Include a precise description of what the beachball represents.

\end{enumerate}

\subsection*{Challenge problem}

Here is some background from \citet{TapeTape2013}. The symbols \xrot{\xi}, \yrot{\xi}, and \zrot{\xi} denote the rotations through angle $\xi$ about the coordinate axes. Thus
%
\begin{equation*}
\xrot{\xi}=\begin{small}\begin{pmatrix}   1 &     0   & 0\\
                                          0 & \cos\xi &-\sin\xi\\
                                          0 & \sin\xi & \cos\xi\end{pmatrix}\end{small},\:\:\:
\yrot{\xi}=\begin{small}\begin{pmatrix}\cos\xi & 0 &\sin\xi \\
                                       0       & 1 & 0      \\
                                      -\sin\xi & 0 & \cos\xi\end{pmatrix}\end{small},\:\:
\zrot{\xi}=\begin{small}\begin{pmatrix}   \cos\xi & -\sin\xi & 0 \\
                                          \sin\xi & \cos\xi & 0 \\
                                          0  & 0 & 1\end{pmatrix}\end{small}.\:\:\:
%\label{eq:DefXxiYxiZxi}
\end{equation*}
%
For $\xi = \pi$ these are
%
\begin{equation*}
\xrot{\pi}=\begin{small}\begin{pmatrix}   1 &     0   & 0\\
                                          0 & -1 & 0\\
                                          0 & 0 & -1\end{pmatrix}\end{small},\:\:\:
\yrot{\pi}=\begin{small}\begin{pmatrix}-1 & 0 & 0 \\
                                       0 & 1 & 0      \\
                                       0 & 0 & -1\end{pmatrix}\end{small},\:\:\;
\zrot{\pi}=\begin{small}\begin{pmatrix}   -1 & 0 & 0 \\
                                          0 & -1 & 0 \\
                                          0  & 0 & 1\end{pmatrix}\end{small}.\:\:\:
\label{eq:DefXxiYxiZxi}
\end{equation*}
%
Use the notation $M_U=UMU^{-1}$, where $U$ is a rotation matrix. Figure~\ref{fig:XXandYY} shows a convention of using two angles, $\xx$ and $\yy$, to characterize the orientation of a beachball.

Verify Eqs.~S3 of the Supporting Information for \citet{TapeTape2013}:
%
\begin{eqnarray}
D(\xx+\pi,\,\yy)&=&\left(D(\xx,\,\yy)\right)_U\hspace{20pt}(U=\xrot{\pi})
\label{eq1}
%\label{eq:DPeriodForXX}
\\
D(\xx,\,\yy+\pi)&=&D(\xx,\,\yy)
\label{eq2}
\\
D(-\xx,\,\yy)&=&\left(D(\xx,\,\yy)\right)_U\hspace{20pt}(U=\yrot{\pi})
\label{eq3}
%\label{eq:DconjugateII}
\\
D(\xx,\,\pi/2-\yy)&=&\left(D(\xx,\,\yy)\right)_U\hspace{20pt}(U=\xrot{\pi})
\label{eq4}
%\label{eq:DconjugateI}
\end{eqnarray}
%
Do this with a partner: one of you does the expression on the left-hand side, the other does the right-hand side. You will need to both start in some reference orientation and apply the same \xx\ and \yy\; I recommend using $\xx=\yy=\dg{0}$ for starters and then trying $\xx=\yy=\dg{45}$.

%------------------------

\begin{figure}[h]
\center
\includegraphics[width=.95\linewidth]{/home/carltape/latex/notes/cmt/figures_cdc/XXandYY_2013-07-14}
\caption{
\citep[Figure~8 of][]{TapeTape2013}. The tensor $D(\xx,\,\yy)$.
%The square is the plane of the crack for the crack tensor.
(a)~$\xx=\yy=0$. This beachball is the double couple $D=D(0,0)$ in a reference orientation.
(b)~$\xx=\pi/4$ and $\yy=0$. This beachball is the result of rotating the first ball through angle \xx\ about its slip vector~\fvecs.
(c)~$\xx=\yy=\pi/4$. This ball is the result of rotating the second ball through angle \yy\ about its neutral vector \fvecb.
\label{fig:XXandYY}
}
\end{figure}

%-------------------------------------------------------------
%\pagebreak
\bibliography{uaf_abbrev,uaf_main,uaf_carletal}
%-------------------------------------------------------------

%-------------------------------------------------------------
\end{document}
%-------------------------------------------------------------
