% dvips -t letter lab_sumatra.dvi -o lab_sumatra.ps ; ps2pdf lab_sumatra.ps
\documentclass[11pt,titlepage,fleqn]{article}

\usepackage{amsmath}
\usepackage{amssymb}
\usepackage{latexsym}
\usepackage[round]{natbib}
%\usepackage{epsfig}
\usepackage{graphicx}
\usepackage{bm}

\usepackage{url}
\usepackage{color}
%\usepackage{hyperref}

%--------------------------------------------------------------
%       SPACING COMMANDS (Latex Companion, p. 52)
%--------------------------------------------------------------

\usepackage{setspace}    % double-space or single-space
\usepackage{xspace}

\renewcommand{\baselinestretch}{1.2}

\textwidth 460pt
\textheight 690pt
\oddsidemargin 0pt
\evensidemargin 0pt

% see Latex Companion, p. 85
\voffset     -50pt
\topmargin     0pt
\headsep      20pt
\headheight   15pt
\headheight    0pt
\footskip     30pt
\hoffset       0pt

\input{commands_letters}
\input{commands_uaf}
\input{commands_carl}

\newcommand{\blank}{xxxx}

\newcommand{\cyear}{2026}

% provide space for students to write their solutions
\newcommand{\vertgap}{\vspace{1cm}}

\graphicspath{
  {./figures/}
}

\newcommand{\cltag}{GEOS 626/426: Applied Seismology, Carl Tape}
\newcommand{\ptag}{{\bf \textcolor{magenta}{[GEOS 626]}}}
%\newcommand{\ptag}{}

\bibliographystyle{agufull08}

\newcommand{\howmuchtime}{Approximately how much time {\em outside of class and lab time} did you spend on this problem set? Feel free to suggest improvements here.}

%------------------------------------------------------

% modes
\newcommand{\tnl}[2]{\mbox{$_{#1}\ssT_{#2}$}}
\newcommand{\snl}[2]{\mbox{$_{#1}\ssS_{#2}$}}
\newcommand{\snlm}[3]{\mbox{$_{#1}\ssS_{#2}^{#3}$}}
\newcommand{\Tnl}{\mbox{${}_nT_l$}}   % eigenfun
\newcommand{\Wnl}{\mbox{${}_nW_l$}}   % eigenfun
\newcommand{\Ynl}{\mbox{${}_nY_l$}}   % spherical harmonics
% eigenfrequencies
\newcommand{\omnl}{\mbox{${}_{n\hspace{-0.2 mm}}\omega_l$}}
\newcommand{\fnl}{\mbox{${}_{n\hspace{-0.2 mm}}f_l$}}
\newcommand{\omnlm}[1]{\mbox{${}_{n\hspace{-0.2 mm}}\omega_l^{#1}$}}
\newcommand{\fnlm}[1]{\mbox{${}_{n\hspace{-0.2 mm}}f_l^{#1}$}}

\newcommand{\cutoff}[1]{{#1}_{x\bar{n}}}
\newcommand{\Wcol}{\textcolor{blue}{W}}
\newcommand{\Tcol}{\textcolor{red}{T}}

%\newcommand{\rfind}{{\tt fzero}}
\newcommand{\rfind}{{\tt brentq}}
%\newcommand{\sfind}{{\tt ode45}}
\newcommand{\sfind}{{\tt solve\_ivp}}
\newcommand{\maxstep}{{\tt max\_step}}

%------------------------------------------------------


% change the figures to ``Figure L3'', etc
\renewcommand{\thefigure}{L\arabic{figure}}
\renewcommand{\thetable}{L\arabic{table}}
\renewcommand{\theequation}{L\arabic{equation}}
\renewcommand{\thesection}{L\arabic{section}}

\newcommand{\tfile}{{\tt lab\_spherical\_harmonics.ipynb}}

\renewcommand{\baselinestretch}{1.0}

%--------------------------------------------------------------
\begin{document}
%-------------------------------------------------------------

\begin{spacing}{1.2} 
\centering
{\large \bf Lab Exercise: Spherical harmonic functions [spherical harmonics]} \\
\cltag\ \\
Last compiled: \today
\end{spacing}
%------------------------

\subsection*{Overview and Instructions}

\begin{itemize}
\item Normal modes of the Earth are mathematically described in terms of spherical harmonic functions.

\item See Section~2.9 of \citet{SteinWysession} for background on normal modes.

\item Check out the Earth's modes via this website by Lucian Saviot:

\url{http://lucien.saviot.free.fr/terre/index.en.html}

These show the motion of the modes at the surface.

\item Check out the Earth's modes via this website by Ruedi Widmer--Schnidrig:

\url{http://www.black-forest-observatory.de/Old_Stuttgart_Webpages/modes.html}

These show the depth eigenfunctions of the modes.

\end{itemize}

%------------------------

\begin{figure}[h]
\centering
\begin{tabular}{cc}
\includegraphics[width=11cm]{surf_0S0.png} \\
\includegraphics[width=11cm]{fig4v_0S0.jpg} 
\end{tabular}
\caption[]
{{
Surface motion and depth eigenfunction for the spheroidal mode \snl{0}{0} ($n=0$, $l=0$).
}}
\label{fig:ex}
\end{figure}

%------------------------

\section*{Questions}

\begin{enumerate}

\item From the modes-surface website, examine the motion for \tnl{0}{2} and \tnl{1}{2}. These modes are featured in \verb+hw_modesA.ipynb+. Use the 3D-cut view setting, as shown in \refFig{fig:ex}.

How would you describe the motion of each of these modes? What do they all have in common?

\vspace{1cm}

\item From the modes-depth website, examine the eigenfunctions for \tnl{0}{2} and \tnl{1}{2}, and \tnl{2}{2}. To plot the eigenfunctions, you need to choose an Earth model, since the eigenfunctions depend on the structure (like $\vp(r)$, $\vs(r)$, and $\rho(r)$, which vary as a function of the Earth's radial coordinate $r$). Choose PREM \cite{PREM}.

How would you describe how each of these modes is sensitive to the Earth's structure?

\vspace{1cm}

\item Identify four prominent modes in the Canberra spectrum of \citet[][Figure~1]{Park2005}. From the modes-surface website, examine their motions.

How would you characterize the motion of these modes? (Hint: Use the words ``breathing'' and ``football'' in your response.) What happens for $l$ increasing from 2 to 3 to 4?

\vspace{1cm}

\item Run \tfile\ for all the examples; the first output figure is shown in \refFig{fig:ylm}.

What is the relationship between $(l,m)$ and the pattern of the $Y_{lm}(\phi,\theta)$ function?

\vspace{1cm}

\item Which examples represent functions that could be used to describe toroidal modes?

\vspace{1cm}

\end{enumerate}

%-------------------------------------------------------------
%\pagebreak
\bibliography{uaf_abbrev,uaf_carletal,uaf_main,uaf_slab}
%-------------------------------------------------------------

\begin{figure}[h]
\hspace{-1cm}
\includegraphics[width=18cm]{lab_ylm.jpg}
\caption[]
{{
Default figure from \tfile.
}}
\label{fig:ylm}
\end{figure}

%-------------------------------------------------------------
\end{document}
%-------------------------------------------------------------
