% dvips -t letter hw_matlab.dvi -o hw_matlab.ps ; ps2pdf hw_matlab.ps
\documentclass[11pt,titlepage,fleqn]{article}

\usepackage{amsmath}
\usepackage{amssymb}
\usepackage{latexsym}
\usepackage[round]{natbib}
%\usepackage{epsfig}
\usepackage{graphicx}
\usepackage{bm}

\usepackage{url}
\usepackage{color}
%\usepackage{hyperref}

%--------------------------------------------------------------
%       SPACING COMMANDS (Latex Companion, p. 52)
%--------------------------------------------------------------

\usepackage{setspace}    % double-space or single-space
\usepackage{xspace}

\renewcommand{\baselinestretch}{1.2}

\textwidth 460pt
\textheight 690pt
\oddsidemargin 0pt
\evensidemargin 0pt

% see Latex Companion, p. 85
\voffset     -50pt
\topmargin     0pt
\headsep      20pt
\headheight   15pt
\headheight    0pt
\footskip     30pt
\hoffset       0pt

\input{commands_letters}
\input{commands_uaf}
\input{commands_carl}

\newcommand{\blank}{xxxx}

\newcommand{\cyear}{2026}

% provide space for students to write their solutions
\newcommand{\vertgap}{\vspace{1cm}}

\graphicspath{
  {./figures/}
}

\newcommand{\cltag}{GEOS 626/426: Applied Seismology, Carl Tape}
\newcommand{\ptag}{{\bf \textcolor{magenta}{[GEOS 626]}}}
%\newcommand{\ptag}{}

\bibliographystyle{agufull08}

\newcommand{\howmuchtime}{Approximately how much time {\em outside of class and lab time} did you spend on this problem set? Feel free to suggest improvements here.}

%------------------------------------------------------

% modes
\newcommand{\tnl}[2]{\mbox{$_{#1}\ssT_{#2}$}}
\newcommand{\snl}[2]{\mbox{$_{#1}\ssS_{#2}$}}
\newcommand{\snlm}[3]{\mbox{$_{#1}\ssS_{#2}^{#3}$}}
\newcommand{\Tnl}{\mbox{${}_nT_l$}}   % eigenfun
\newcommand{\Wnl}{\mbox{${}_nW_l$}}   % eigenfun
\newcommand{\Ynl}{\mbox{${}_nY_l$}}   % spherical harmonics
% eigenfrequencies
\newcommand{\omnl}{\mbox{${}_{n\hspace{-0.2 mm}}\omega_l$}}
\newcommand{\fnl}{\mbox{${}_{n\hspace{-0.2 mm}}f_l$}}
\newcommand{\omnlm}[1]{\mbox{${}_{n\hspace{-0.2 mm}}\omega_l^{#1}$}}
\newcommand{\fnlm}[1]{\mbox{${}_{n\hspace{-0.2 mm}}f_l^{#1}$}}

\newcommand{\cutoff}[1]{{#1}_{x\bar{n}}}
\newcommand{\Wcol}{\textcolor{blue}{W}}
\newcommand{\Tcol}{\textcolor{red}{T}}

%\newcommand{\rfind}{{\tt fzero}}
\newcommand{\rfind}{{\tt brentq}}
%\newcommand{\sfind}{{\tt ode45}}
\newcommand{\sfind}{{\tt solve\_ivp}}
\newcommand{\maxstep}{{\tt max\_step}}

%------------------------------------------------------


\renewcommand{\baselinestretch}{1.1}

\newcommand{\tfile}{{\tt hw\_gr.ipynb}}

\newcommand{\fmag}{frequency--magnitude}

%--------------------------------------------------------------
\begin{document}
%-------------------------------------------------------------

\begin{spacing}{1.2}
\centering
{\large \bf Problem Set 1: \\
The Gutenberg--Richter \fmag\ relation [gr]} \\
\cltag\ \\
Assigned: January 17, \cyear\ --- Due: January 26, \cyear\ \\
Last compiled: \today
\end{spacing}

%------------------------

\subsection*{Overview and instructions}

\begin{itemize}

\item The purposes of this problem set are:
%
\begin{enumerate}
\item to practice using Python for scientific computing and plotting
\item to think critically about histograms
\item to review and apply some concepts of differentiation
\item to understand the \fmag\ relation (\refeq{GR})\footnote{This empirical formula is referred to as the combined phrase a + b, where a = ``\fmag'' or ``Gutenberg--Richter'' and b = ``relation'' or ``relationship'' or ``distribution'' or ``law''.}, including some of its subtler points
\end{enumerate}
%
If you are new to Python, let me know.

\item Suggested reading:
\begin{itemize}
\item \citet[][Section 4.7.1]{SteinWysession}
\item \citet{GutenbergRichter1944}
\end{itemize}

%\item See the handout \verb+doc_startup.pdf+ to get set up on the Linux network. This includes copying the template file: \\
%\verb+cp hw_gr_template.m hw_gr.m+
%to access \verb+GEOTOOLS+.

\item Log into OpenScienceLab and run the template notebook \tfile. You can work directly in this file.

\item {\bf Possible source of confusion:} Following standard mathematical notation, I will use `$\log_{10}$' to represent the base-10 logarithm and `$\ln$' to represent the natural logarithm. Note that in Python \verb+log10+ is $\log_{10}$ and \verb+log+ is $\ln$. (Note that \citet{SteinWysession} use `$\log$' for the base-10 logarithm.) Reminder: $\log_{10} x = \ln x / \ln 10$

\item {\bf Python tips:}
%Use \verb+linspace+ to generate a sequency of numbers, $x_i$; then $10^{x_i}$ will be uniformly spaced in $\log$ space.
\tfile\ shows an example of how to plot multiple items on a log-scaled plot.
%It also shows how to produce a postscript (PS) or pdf file from a Matlab file.
Make sure you follow the template script before proceeding.

\item {\bf Written responses:}
Complete as many answers as possible from inside the Jupyter notebook. Other responses, such as equation derivations (if needed), can be done with pencil and paper, then scanned and uploaded to the google drive.

\item {\bf Latex option:}
If you want to use Latex for your homeworks, see the templates in the folder \verb+latex+ (and see the \verb+README+).

\end{itemize}

%------------------------

\pagebreak
\subsection*{The Gutenberg--Richter \fmag\ relation}

The Gutenberg--Richter \fmag\ relation \citep{GutenbergRichter1944} is one of the cornerstone empirical relationships in seismology. It is given by
%
\begin{equation}
\log_{10} N = a - b\,M,
\label{GR}
\end{equation}
%
where
%
\begin{itemize}
\item $N$ is the cumulative number of earthquakes having magnitudes larger than $M$ that occur in region $R$ within a particular time $T$
\item $M$ is the earthquake magnitude; we take this to be moment magnitude $\mw$
\item $b$ controls the slope of the seismicity distribution in region $R$ within a particular time $T$. The line will always increase to the left, so \refEq{GR} has a negative slope and therefore $b > 0$ by definition.
\item $b \approx 1$ for most earthquake catalogs
\item $a$ indicates the seismic activity in region $R$ within a particular time $T$
\end{itemize}
%
A subtle point is that in practice the magnitudes are {\em binned}, then a line is fit to the histogram. It is helpful to consider the discrete form of \refEq{GR}:
%
\begin{equation}
\log_{10} N_i = a - b\,M_i,
\label{GRd}
\end{equation}
%
where the index $i$ refers to the magnitude bin, with $M_i$ being the magnitude at the left boundary of the bin.

\begin{itemize}
\item The {\em cumulative distribution} is the cumulative number of events with $M \ge M_i$; the \fmag\ relation is expressed as a cumulative distribution.
\item The {\em incremental distribution} is the number of events per magnitude bin, $M_i \le M < M_{i+1}$. (Here we are adoping the convention that the uppermost value $M_{i+1}$ is excluded from the interval. Check the details of your histogram-making function.)
\item A {\em magnitude interval} is a range of magnitude, \eg the magnitude interval $[8.7,9.0)$. For the cumulative distribution, a magnitude interval should be expressed as an open interval, such as $M \ge 6.2$, rather than $6.2 \le M < 6.3$.
\end{itemize}

%\pagebreak

%------------------------

\pagebreak
\subsection*{Problem 1 (10.0). Histograms and earthquake statistics}

{\bf Make sure that you are very comfortable with the meaning of \refEq{GR} before proceeding.}

\begin{enumerate}
\item (0.2) We consider the Global Centroid Moment Tensor (GCMT) catalog (\verb+www.globalcmt.org+), from 01-Jan-1976 to 30-June-2011. Thus $T$ represents the duration of the catalog, and $R$ represents planet Earth.

Run the Python notebook \tfile\ to generate a global map of the catalog.
%
\begin{enumerate}
\item (0.0) What is the range of depths of events in the catalog?
\item (0.1) List three regions of the deepest seismicity (\eg Tonga--Kermadec).
\item (0.1) What is the smallest magnitude and largest magnitude in the catalog? List your magnitudes with 0.001 precision.
\end{enumerate}

\item (0.8) Running \tfile\ will call the function \verb+seis2GR.py+ to obtain the cumulative and incremental distributions for the $\mw$ values of the GCMT catalog, using a bin width of $\Delta M = 0.1$. Examine the output that appears.

%Note: The function \verb+seis2GR.py+ uses the variables names \verb+N+ for the cumulative numbers $N_i$, \verb+Ninc+ for the incremental numbers, and \verb+Medges+ for the bin edges $M_i$.

\begin{enumerate}
\item (0.2) What is the maximum value of the incremental distribution?

What is its magnitude interval?

\item (0.2) What is the maximum value of the cumulative distribution?

What is its magnitude interval?

\item (0.2) What is the minimum value of the incremental distribution over the interval $[4.2,9.1]$? 

What is its magnitude interval?

\item (0.2) What is the minimum value of the cumulative distribution over the interval $[4.2,9.1]$? 

What is its magnitude interval?

\end{enumerate}

%----------------

\item (3.0) Now it's time to write some lines of code.

\begin{enumerate}
\item (0.0) Using the output from \verb+seis2GR.py+ (as shown in \tfile), {\bf plot the cumulative and incremental distributions on the same plot}, similar to the plot in \citet[][Figure~4.7-2]{SteinWysession}, but note that your $x$-axis is $\mw$, not $\log_{10} M_0$, and your $y$-axis is number of earthquakes, not number of earthquakes per year.

{\em Coding tip}: There are two ways to deal with log scaling. One way is to transform $N$ into $n = \log_{10} N$, then work with~$n$ and use the \verb+plot+ command. Another way is to work with $N$ directly and use the \verb+semilogy+ command, as shown in the example at the bottom of \tfile.
%(Pay attention to how the \verb+hold+ command works; this can be tricky.)

Which distribution---incremental or cumulative---has more scatter?

\pagebreak

\item (2.2)
\begin{enumerate}
\item (0.6) Find a best-fitting line, $\log_{10} N = a - b M$, for the `most linear' section of the $\log_{10}$-scaled cumulative distribution. You can simply pick two points and compute the line, or use a command such as \verb+polyfit+ (and \verb+polyval+) to apply a least-squares fit to a set of points. (Do not fit a line to the entire distribution!) What are your values for $a$ and $b$? (Show your work.)

\item (1.0) Plot your best-fitting line along with the full data set.
Include both the incremental and cumulative distributions in your plot.

\item (0.3) $a$ is the $y$-intercept. What is its physical meaning?

\item (0.3) $b$ is the slope. What is its physical meaning?

\end{enumerate}

%-------

\item (0.6) Assume that the best-fitting distribution (not the GCMT catalog) is `reality'. Based on the idealized cumulative distribution, what is largest earthquake expected over the duration of the GCMT catalog? (Show your work.)

Hint: Where does $N = 1$ intersect your best-fitting line?

Note: The expected value does not have to agree with what actually occurred within the GCMT catalog.

%-------

\item (0.2) The ``catalog completeness'' \citep[\eg][]{WiemerWyss2000} $M_c$ represents the smallest magnitude above which the \fmag\ distribution is representative for a particular seismicity catalog. What is the catalog completeness for GCMT? List your answer with 0.1 precision. (Provide a brief explanation, but no computation is necessary.)

\end{enumerate}

%--------------------------------------

\item (1.0) Instead of analyzing seismicity, let us now analyze seismicity rate by dividing all binned values by the duration of the catalog ($T$, in years).
%
\begin{enumerate}
\item (0.1) Considering the total duration and the total number of events in the GCMT catalog, 
%Taking an average over the entire time interval of the GCMT catalog,
how many earthquakes per year are there?

\item (0.1) Why is seismicity rate more useful than seismicity?

\item (0.5) What is the best-fitting line (namely, $a$ and $b$) for the new distribution?
You can determine this graphically or analytically, from your previous results.
(Show your work.)

\item (0.1) What magnitude interval averages $>$100 events per year?

\item (0.1) How many $M \ge 0$ earthquakes are expected on Earth per year?

\item (0.1) How many earthquakes (of any magnitude) are expected on Earth per year?
\end{enumerate}

%----------------

\item (1.0)
%
\begin{enumerate}
\item (0.8) Plot the cumulative and incremental distributions for seismicity rate for bin widths of $\Delta M = 0.05, 0.10, 0.5$, and 1.0.
\item (0.2) What is the apparent relationship between bin width and the separation between the cumulative and incremental distributions?
\end{enumerate}

%----------------

\pagebreak
\item (2.0) ({\em A computer is not needed for this problem.}) \\
You will now try to show mathematically what you identified graphically in 5b. \\
Define the bin width as
%
\begin{equation}
\Delta M = M_{i+1} - M_i
\label{dM}
\end{equation}
%
where $i$ increases to the right (the usual convention).

The incremental distribution is given by
%
\begin{equation}
-\Delta N_i = -(N_{i+1} - N_i) = N_i - N_{i+1}.
\end{equation}
%
\begin{enumerate}
\item (1.7) Using the discrete \fmag\ relation (\refeq{GRd}), show that the incremental distribution can be written as
%
\begin{equation}
\log_{10}(-\Delta N_i) = a + \Delta a - b M_i
\label{inc}
\end{equation}
%
where $\Delta a$ is a function in terms of other variables (but not $i$). \\
List the expression for $\Delta a$.

%Hint: Start with $\log_{10}(-\Delta N_i)$ and show that it can be written in the form of \refEq{inc}.

\item (0.1) The equation answering the previous problem provides a mathematical relationship between bin width and the shift in the $y$-intercept. Describe in words what happens for increasing $\Delta M$.

\item (0.2) If $b = 1$ and $\Delta M = 0.1$, what is $\Delta a$?
\end{enumerate}

%----------------

\item  (1.5) \ptag\ ({\em A computer is not needed for this problem.})
%As suggested in \citet[][p.~274]{SteinWysession}, the incremental distribution is related to the derivative of the cumulative distribution. 
%
\begin{enumerate}
\item (0.3) Differentiate \refEq{GR} to obtain $d N/ d M$.

%Starting from \refEq{GR}, derive the analytical formula for $d N/ d M$.

\item (1.0) Using your expression from (a), as well as the mathematical definition of a derivative, derive an expression analagous to \refEq{inc} that is valid for small bin widths, $\Delta M \ll 1$. List the expression for $\Delta a$

\item (0.2) If $b = 1$ and $\Delta M = 0.1$, what is $\Delta a$?

\end{enumerate}

%----------------

\item (0.5) Earlier you determined the catalog completeness, $M_c$.
%
\begin{enumerate}
\item (0.1) Can the GCMT catalog, $M > M_c$ (see earlier part of this problem for $M_c$), be fit well with a single line?
\item (0.2) Compute an estimate for $b$ for $M > 7.5$.
\item (0.1) What does the different $b$ value imply about large events in the catalog?
\item (0.1) What is a possible reason for this?
\end{enumerate}

\end{enumerate}

%-------------------------------------------------------------

\pagebreak
\subsection*{Problem} \howmuchtime\

%-------------------------------------------------------------
%\pagebreak
\bibliography{uaf_abbrev,uaf_main,uaf_source,uaf_calif,uaf_carletal,uaf_imaging,uaf_alaska}
%-------------------------------------------------------------

\iffalse
\begin{figure}[h]
\hspace{-1cm}
\includegraphics[width=15cm]{gcmt_catalog.eps}
\caption[]
{{
GCMT catalog.
\label{fig:gcmt}
}}
\end{figure}
\fi

%-------------------------------------------------------------
\end{document}
%-------------------------------------------------------------
