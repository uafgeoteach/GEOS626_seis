% dvips -t letter lab_mt.dvi -o lab_mt.ps ; ps2pdf lab_mt.ps
\documentclass[11pt,titlepage,fleqn]{article}

\usepackage{amsmath}
\usepackage{amssymb}
\usepackage{latexsym}
\usepackage[round]{natbib}
%\usepackage{epsfig}
\usepackage{graphicx}
\usepackage{bm}

\usepackage{url}
\usepackage{color}
%\usepackage{hyperref}

%--------------------------------------------------------------
%       SPACING COMMANDS (Latex Companion, p. 52)
%--------------------------------------------------------------

\usepackage{setspace}    % double-space or single-space
\usepackage{xspace}

\renewcommand{\baselinestretch}{1.2}

\textwidth 460pt
\textheight 690pt
\oddsidemargin 0pt
\evensidemargin 0pt

% see Latex Companion, p. 85
\voffset     -50pt
\topmargin     0pt
\headsep      20pt
\headheight   15pt
\headheight    0pt
\footskip     30pt
\hoffset       0pt

\input{commands_letters}
\input{commands_uaf}
\input{commands_carl}

\newcommand{\blank}{xxxx}

\newcommand{\cyear}{2026}

% provide space for students to write their solutions
\newcommand{\vertgap}{\vspace{1cm}}

\graphicspath{
  {./figures/}
}

\newcommand{\cltag}{GEOS 626/426: Applied Seismology, Carl Tape}
\newcommand{\ptag}{{\bf \textcolor{magenta}{[GEOS 626]}}}
%\newcommand{\ptag}{}

\bibliographystyle{agufull08}

\newcommand{\howmuchtime}{Approximately how much time {\em outside of class and lab time} did you spend on this problem set? Feel free to suggest improvements here.}

%------------------------------------------------------

% modes
\newcommand{\tnl}[2]{\mbox{$_{#1}\ssT_{#2}$}}
\newcommand{\snl}[2]{\mbox{$_{#1}\ssS_{#2}$}}
\newcommand{\snlm}[3]{\mbox{$_{#1}\ssS_{#2}^{#3}$}}
\newcommand{\Tnl}{\mbox{${}_nT_l$}}   % eigenfun
\newcommand{\Wnl}{\mbox{${}_nW_l$}}   % eigenfun
\newcommand{\Ynl}{\mbox{${}_nY_l$}}   % spherical harmonics
% eigenfrequencies
\newcommand{\omnl}{\mbox{${}_{n\hspace{-0.2 mm}}\omega_l$}}
\newcommand{\fnl}{\mbox{${}_{n\hspace{-0.2 mm}}f_l$}}
\newcommand{\omnlm}[1]{\mbox{${}_{n\hspace{-0.2 mm}}\omega_l^{#1}$}}
\newcommand{\fnlm}[1]{\mbox{${}_{n\hspace{-0.2 mm}}f_l^{#1}$}}

\newcommand{\cutoff}[1]{{#1}_{x\bar{n}}}
\newcommand{\Wcol}{\textcolor{blue}{W}}
\newcommand{\Tcol}{\textcolor{red}{T}}

%\newcommand{\rfind}{{\tt fzero}}
\newcommand{\rfind}{{\tt brentq}}
%\newcommand{\sfind}{{\tt ode45}}
\newcommand{\sfind}{{\tt solve\_ivp}}
\newcommand{\maxstep}{{\tt max\_step}}

%------------------------------------------------------


\renewcommand{\baselinestretch}{1.1}

% change the figures to ``Figure L3'', etc
\renewcommand{\thefigure}{L\arabic{figure}}
\renewcommand{\thetable}{L\arabic{table}}
\renewcommand{\theequation}{L\arabic{equation}}
\renewcommand{\thesection}{L\arabic{section}}

% for referencing anything in the supplement
\usepackage{xr}
\externaldocument{/home/carltape/latex/notes/cmt/TapeTape/notes_mt_626}

\include{waltcommands}

\newcommand{\mtfile}{\texttt{notes\_mt\_626.pdf}}

\graphicspath{
  {/home/carltape/latex/notes/cmt/TapeTape/figures/}
}

\newcommand{\fvect}{\textcolor{red}{\mbT}}
\newcommand{\fvecb}{\textcolor{blue}{\mbB}}
\newcommand{\fvecp}{\textcolor{black}{\mbP}}
\newcommand{\fvecn}{\mbN}
\newcommand{\fvecs}{\mbS}

\newcommand{\eone}{\textcolor{red}{\mbe_1}}
\newcommand{\etwo}{\textcolor{blue}{\mbe_2}}

%--------------------------------------------------------------
\begin{document}
%-------------------------------------------------------------

\begin{spacing}{1.2}
\centering
{\large \bf Lab Exercise: Moment tensor as a linear transformation [mt]} \\
\cltag\ \\
%Assigned: February 9, 2012 --- Due: February 21, 2012
Last compiled: \today
\end{spacing}

% assume that Figures 1-7 are in beachnotes_626.pdf
%\setcounter{figure}{7}
%\setcounter{table}{1}
%------------------------

\subsection*{Instructions}

This is a pencil-and-paper lab exercise that does not involve any computer. You should have a ruler, and you are permitted to use a calculator. The exercise directly follows the class notes \mtfile, which has Figures 1--7 and Table 1. I recommend using colored pens or pencils (especially red and blue).

\medskip\noindent
{\em Note}: When estimating the coordinates (\ie the lengths of each component) of a vector, you want to imagine the origin to be at the tail of the vector. 

%------------------------

\subsection*{Exercises}

\begin{enumerate}

\item Examine Figure~\ref{fig:template} and derive the equations in the figure.
\label{prob:bigvec}

\begin{enumerate}
\item Figure~5 in the notes showed a vector field for a double couple moment tensor. Does the vector field in Figure~\ref{fig:template} double couple moment tensor? Why or why not?

\item Label the following vectors: $\mb{M}(-\eone)$ and $\mb{M}(-\etwo)$.

Visually check that $\mb{M}(-\eone) = -\mb{M}(\eone)$ and $\mb{M}(-\etwo) = -\mb{M}(\etwo)$.

\item Draw and label the following vectors: \textcolor{red}{$m_{11}\mbe_1$}, \textcolor{red}{$m_{21}\mbe_2$}, \textcolor{blue}{$m_{12}\mbe_1$}, \textcolor{blue}{$m_{22}\mbe_2$}.

Draw the base of each vector at the tip of $\eone$, $-\eone$, $\etwo$, or $-\etwo$.

Draw and label the following vectors: \textcolor{red}{$-m_{11}\mbe_1$}, \textcolor{red}{$-m_{21}\mbe_2$}, \textcolor{blue}{$-m_{12}\mbe_1$}, \textcolor{blue}{$-m_{22}\mbe_2$}.

\item Measure the radius of the circle in mm. You will use this to calibrate the other measurements.

\item Assuming that $\|\eone\| = \|\etwo\| = 1$, use the sketch to estimate the entries of the moment matrix
%
\begin{equation*}
M = \begin{pmatrix} \textcolor{red}{m_{11}} & \textcolor{blue}{m_{12}} \\ \textcolor{red}{m_{21}} & \textcolor{blue}{m_{22}} \end{pmatrix}
\end{equation*}
%
for this choice of basis.
(Check your answer using Table~1 of \mtfile.)

\item With two dashed lines (like spokes of a wheel), mark the principal axes basis.

\end{enumerate}

%--------------

\item 
\label{prob:matrix}
We now want to relate our graphical representation of the moment matrix with some standard depictions in the literature. One is that the moment matrix entries are represented as couples, such as in \citet[][Figure 4.4-4]{SteinWysession} (Figure~\ref{fig:SW} here), \citet[][Figure~9.2]{ShearerE2}, and \citet[][Figure 3.7]{AkiRichardsE2}.
%
\begin{enumerate}
\item Sketch the force couples in Figure~\ref{fig:template4} associated with the four entries of the matrix of the moment tensor $m_{ij}$. You should sketch 8 vectors in total---the same number you sketched in Figure~\ref{fig:template}.
(Here we have a $2 \times 2$ matrix $m_{ij}$, not a $3 \times 3$ matrix, but the basic concepts are the same.)

Draw the base of each vector at the tip of $\eone$, $-\eone$, $\etwo$, or $-\etwo$.

Color each vector red or blue, depending on whether it is part of \textcolor{red}{$\bM(\be_1)$} or \textcolor{blue}{$\bM(\be_2)$}.

\item Normalize your 8 vectors and superimpose them as black unit vectors. This will provide a more direct analog for the 3D version shown in \refFig{fig:SW}.

\item Which entries of the matrix represent the classical description of the double couple?
%Label the vectors that comprise the double couple.
(See ``Nomenclature confusion'' in \mtfile.)
\end{enumerate}

%--------------

\item 
\label{prob:couples1}
See Figure~\ref{fig:template_couples1}.
Write the moment matrix for each of the three examples.

%--------------

\item
\label{prob:couples2}
See Figure~\ref{fig:template_couples2}.
%See Figure~\ref{fig:couples2} of \mtfile.
%
\begin{enumerate}
\item Write the moment matrix for the center figure.
\item What is odd about this pair of couples?
\item Sketch the standard basis ($\eone$ pointing to the right; $\etwo$ pointing up) and associated colored vectors in the right-hand template. What is the moment matrix?
\item Write the moment matrix for the left-hand figure.
\end{enumerate}

%--------------

\item 
\label{prob:DC}
See Figure~\ref{fig:templateDC}.
%
\begin{enumerate}
\item Write the matrix of the moment tensor for each of the vector fields.

\item The $\fvect$-axis (``tensional'' axis) is the eigenvector direction associated with positive eigenvalue. The $\fvecp$-axis (``compressional'' axis) is the eigenvector direction associated with negative eigenvalue. Label $\fvect$ and $\fvecp$ in the right-hand plot. (You have multiple options.)

\item The fault normal vector is $\fvecn$. The fault slip vector is $\fvecs$.
The eigenvectors $\fvect$ and $\fvecp$ are not the fault vectors. For a double couple moment tensor only, the directions of the fault and slip vectors can be obtained from the eigenvectors as\footnote{This is not the only choice for defining $\fvecn$ and $\fvecs$.}
%
\begin{eqnarray*}
\fvecn = (\fvect + \fvecp)/\sqrt{2}
\\
\fvecs = (\fvect - \fvecp)/\sqrt{2}
\end{eqnarray*}
%
Sketch $\fvecn$ and $\fvecs$ at the application point $\eone$ in the right-hand plot.

\item What do the basis vectors in the left-hand plot represent?

\item In the left-hand plot, shade the two quadrants containing the $T$ axis, and sketch the double couple (four arrows) at the center of the plot.

\item Based on (c), make the fault plane {\bf bold}. If this is map view, what kind of fault is this?

\item If instead we swap $\fvecn$ and $\fvecs$, then what type of fault is it?
\end{enumerate}

%--------------

\item 
\label{prob:decom}
Figure~\ref{fig:decom}.
%
\begin{enumerate}
\item Identify the magenta vectors in the bottom two plots, and highlight the same vectors in the corresponding top two plots.
\item Sketch the basis vectors $\{\eone,\etwo\}$ in the top two.
\item Each vector can be decomposed into an ``isotropic'' part that points outward. Check that the isotropic vectors are all the same. Why is this the case?
\end{enumerate}

%--------------

\item 
\label{prob:arrows}
In \refFig{fig:SWfour}, use the templates and draw the couples (in red) for the following moment matrices, shown in the standard basis:
%
\begin{equation*}
%M =
\begin{pmatrix} 3 &  0 & 0 \\  0 & 2 & 0 \\ 0 & 0 &  1 \end{pmatrix},
\begin{pmatrix} -1 &  0 & 0 \\  0 & -2 & 0 \\ 0 & 0 & -3 \end{pmatrix},
\begin{pmatrix} 0 & -1 & 0 \\ -1 & 1 & 0 \\ 0 & 0 & -1 \end{pmatrix},
\begin{pmatrix} 2 &  0 & 1 \\  0 & 0 & 1 \\ 1 & 1 & -2 \end{pmatrix}
\end{equation*}

%--------------

\end{enumerate}

%==============================================================

%\clearpage\pagebreak

% \begin{figure}
% \centering
% \begin{tabular}{cc}
% \includegraphics[width=10cm]{couples1meaningOfM} &
% \includegraphics[width=8cm]{couples1_left}
% \end{tabular}
% \caption{
% Repeat of Figure~\ref{fig:couples3doubleCouple} for class exercise.
% \label{fig:templateDC}
% }
% \end{figure}

\begin{figure}
\centering
\includegraphics[width=\linewidth]{couples_mod}
\caption{
[Problem~\ref{prob:bigvec}] Same as Figure~\ref{fig:couples1} of \mtfile, but here the basis chosen is to be aligned with the edges of the page.
\label{fig:template}
}
\end{figure}


\begin{figure}
\centering
\begin{tabular}{|l|l|}
\hline
\includegraphics[width=0.5\linewidth]{couples_mod_notext_nocolor} & 
\includegraphics[width=0.5\linewidth]{couples_mod_notext_nocolor} \\
$m_{11}$ & $m_{12}$ \\ \hline
\includegraphics[width=0.5\linewidth]{couples_mod_notext_nocolor} &
\includegraphics[width=0.5\linewidth]{couples_mod_notext_nocolor} \\
$m_{21}$ & $m_{22}$ \\ \hline
\end{tabular}
\caption{
[Problem~\ref{prob:matrix}] 
The four panels should be thought of as the entries of the $2 \times 2$ matrix $M$, with entries $m_{11}$, $m_{12}$, $m_{21}$, and $m_{22}$.
Tip: $m_{12}$ will correspond to forces pointing in the $\pm\eone$ directions and offset in the $\pm\etwo$ direction (e.g., see Figure~\ref{fig:SW} for the 3D analog).
\label{fig:template4}
}
\end{figure}

\begin{figure}
\hspace{-1cm}
\includegraphics[width=18cm]{couples1}
\caption{
[Problem~\ref{prob:couples1}] Repeat of Figure~\ref{fig:couples1} from \mtfile.
Make measurements on these figures to determine the entries of the moment tensor matrix for each case.
\label{fig:template_couples1}
}
\end{figure}

\begin{figure}
\hspace{-1cm}
%\includegraphics[width=18cm]{couples2}
\includegraphics[width=18cm]{couples2_template}
\caption{
[Problem~\ref{prob:couples2}] 
Repeat of Figure~\ref{fig:couples2} from \mtfile, but with an extra vector field template for annotating.
\label{fig:template_couples2}
}
\end{figure}

\begin{figure}
\hspace{-1cm}
\includegraphics[width=18cm]{couples3doubleCouple}
\caption{
[Problem~\ref{prob:DC}] 
Repeat of Figure~\ref{fig:couples3doubleCouple} from \mtfile.
\label{fig:templateDC}
}
\end{figure}

\begin{figure}
\centering
\begin{tabular}{ccc}
\includegraphics[height=8cm]{couples1_nobasis}
& \hspace{1.4cm} &
\includegraphics[height=8cm]{couples1_nobasis}
\end{tabular}
\\
\makebox{\hspace{0.5cm}}
\includegraphics[height=8cm]{couples4decomposition_new}
\caption{
[Problem~\ref{prob:decom}]
Repeat of Figures~\ref{fig:couples1} and \ref{fig:couples4decomposition}.
The magenta vectors at the bottom are from the set of black vectors at the top.
In the bottom plots, note that all vectors originate from the tip of $\eone$, $-\eone$, $\etwo$, or $-\etwo$.
No matter what the choice of basis, the isotropic components (green) will be the same magnitude.
%When the basis is chosen to be the principal axes, as at right, the decomposition can be represented simply with eigenvalues.
\label{fig:decom}
}
\end{figure}

%\ref{fig:SingleCouple}
%\ref{fig:couples1meaningOfM}
%\ref{fig:couples1}
%\ref{fig:couples2}
%\ref{fig:couples3doubleCouple}
%\ref{fig:couples4decomposition}

\begin{figure}
\centering
\includegraphics[width=0.8\linewidth]{SW_F4.4-4.eps}
\caption{
[Problem~\ref{prob:arrows}]
Figure 4.4-4 from \citet{SteinWysession}.
After the exercise you should see the connection between Figure~\ref{fig:template4} (which is $2 \times 2$, not $3 \times 3$) and this figure.
\label{fig:SW}
}
\end{figure}

\begin{figure}
\hspace{-1cm}
\begin{tabular}{c|c}
\includegraphics[width=9cm]{SW_F4.4-4.eps} & \includegraphics[width=9cm]{SW_F4.4-4.eps} \\
& \\ \hline
& \\
\includegraphics[width=9cm]{SW_F4.4-4.eps} & \includegraphics[width=9cm]{SW_F4.4-4.eps} 
\end{tabular}
\caption{
[Problem~\ref{prob:arrows}]
\label{fig:SWfour}
}
\end{figure}

%-------------------------------------------------------------
\clearpage\pagebreak
\bibliography{uaf_abbrev,uaf_main,uaf_carletal}
%-------------------------------------------------------------

%====================================================================
\end{document}
%====================================================================

