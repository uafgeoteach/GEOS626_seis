% dvips -t letter lab_tomo.dvi -o lab_tomo.ps ; ps2pdf lab_tomo.ps
\documentclass[11pt,titlepage,fleqn]{article}

\usepackage{amsmath}
\usepackage{amssymb}
\usepackage{latexsym}
\usepackage[round]{natbib}
%\usepackage{epsfig}
\usepackage{graphicx}
\usepackage{bm}

\usepackage{url}
\usepackage{color}
%\usepackage{hyperref}

%--------------------------------------------------------------
%       SPACING COMMANDS (Latex Companion, p. 52)
%--------------------------------------------------------------

\usepackage{setspace}    % double-space or single-space
\usepackage{xspace}

\renewcommand{\baselinestretch}{1.2}

\textwidth 460pt
\textheight 690pt
\oddsidemargin 0pt
\evensidemargin 0pt

% see Latex Companion, p. 85
\voffset     -50pt
\topmargin     0pt
\headsep      20pt
\headheight   15pt
\headheight    0pt
\footskip     30pt
\hoffset       0pt

\input{commands_letters}
\input{commands_uaf}
\input{commands_carl}

\newcommand{\blank}{xxxx}

\newcommand{\cyear}{2026}

% provide space for students to write their solutions
\newcommand{\vertgap}{\vspace{1cm}}

\graphicspath{
  {./figures/}
}

\newcommand{\cltag}{GEOS 626/426: Applied Seismology, Carl Tape}
\newcommand{\ptag}{{\bf \textcolor{magenta}{[GEOS 626]}}}
%\newcommand{\ptag}{}

\bibliographystyle{agufull08}

\newcommand{\howmuchtime}{Approximately how much time {\em outside of class and lab time} did you spend on this problem set? Feel free to suggest improvements here.}

%------------------------------------------------------

% modes
\newcommand{\tnl}[2]{\mbox{$_{#1}\ssT_{#2}$}}
\newcommand{\snl}[2]{\mbox{$_{#1}\ssS_{#2}$}}
\newcommand{\snlm}[3]{\mbox{$_{#1}\ssS_{#2}^{#3}$}}
\newcommand{\Tnl}{\mbox{${}_nT_l$}}   % eigenfun
\newcommand{\Wnl}{\mbox{${}_nW_l$}}   % eigenfun
\newcommand{\Ynl}{\mbox{${}_nY_l$}}   % spherical harmonics
% eigenfrequencies
\newcommand{\omnl}{\mbox{${}_{n\hspace{-0.2 mm}}\omega_l$}}
\newcommand{\fnl}{\mbox{${}_{n\hspace{-0.2 mm}}f_l$}}
\newcommand{\omnlm}[1]{\mbox{${}_{n\hspace{-0.2 mm}}\omega_l^{#1}$}}
\newcommand{\fnlm}[1]{\mbox{${}_{n\hspace{-0.2 mm}}f_l^{#1}$}}

\newcommand{\cutoff}[1]{{#1}_{x\bar{n}}}
\newcommand{\Wcol}{\textcolor{blue}{W}}
\newcommand{\Tcol}{\textcolor{red}{T}}

%\newcommand{\rfind}{{\tt fzero}}
\newcommand{\rfind}{{\tt brentq}}
%\newcommand{\sfind}{{\tt ode45}}
\newcommand{\sfind}{{\tt solve\_ivp}}
\newcommand{\maxstep}{{\tt max\_step}}

%------------------------------------------------------


\newcommand{\tfile}{{\tt lab\_tomo.ipynb}}

% change the figures to ``Figure L3'', etc
\renewcommand{\thefigure}{L\arabic{figure}}
\renewcommand{\thetable}{L\arabic{table}}
\renewcommand{\theequation}{L\arabic{equation}}
\renewcommand{\thesection}{L\arabic{section}}

%--------------------------------------------------------------
\begin{document}
%-------------------------------------------------------------

\begin{spacing}{1.2}
\centering
{\large \bf Lab Exercise: Basis functions for seismic tomography [tomo]} \\
\cltag\ \\
%Assigned: April 10, 2014 --- Due: April 17, 2014 \\
Last compiled: \today
\end{spacing}

%------------------------

\subsection*{Overview}

This lab is preparation for {\tt hw\_tomo.pdf}. The template notebook is \tfile. Here we focus on three concepts: 1) using basis functions to build other functions, 2) discretizing ray paths on the surface of a sphere, and 3) numerical integration.

\subsection*{Questions}

A tomographic model is typically described in terms of a perturbation in wave speed structure from some reference model, \ie $\delta\ln c = \delta c/c$.
(Note \footnote{See the section entitled ``ASIDE: the natural log function used to express tomographic models'' in the {\tt hw\_math} solutions.})
This function may be expanded in terms of a set of $M$ basis functions:
%
\begin{equation}
\delta\ln c(\bx)=\sum_{k=1}^{\nparm} \delta m_k\,B_k(\bx),
\label{dlnv}
\end{equation}
%
where $\delta m_k$, $k=1,\ldots,\nparm$, represent the model coefficients describing the model perturbation $\delta\ln c(\bx)$, and $B_k(\bx)$ is a basis function. Here we choose to use spherical spline basis functions \citep{WangDahlen1995spline,Wang1998}.

\begin{enumerate}

\item Run the Python script \verb+lab_tomo.ipynb+ in order to familiarize yourself with spherical spline basis functions. This script will generate a spherical spline basis function of grid order $q$, centered at a randomly selected gridpoint. Change the value for $q$ and see what happens. 
%An example of a $q = 8$ basis function is shown in \refFig{fig:basis}b.

\item What are the values of the $q=8$ spline function that is centered at (longitude, latitude) = $(-117^\circ,\;34^\circ)$, evaluated at the points $(\phi,\;34^\circ)$, where $\phi = -117, -117.05, \ldots, -118$?

\item In order to construct a matrix of partial derivatives for the tomographic inversion, you will need to know how to integrate functions along ray paths that are on the sphere. 

Discretize a ray path between two points on the sphere.

Use \verb+Geod()+ from the pyproj module.

\item How will you perform numerically integrate a function evaluated along a ray path?

\end{enumerate}

%-------------------------------------------------------------
%\pagebreak
\bibliography{uaf_abbrev,uaf_main,uaf_carletal}
%-------------------------------------------------------------

%-------------------------------------------------------------
\end{document}
%-------------------------------------------------------------
