% dvips -t letter lab_dispersion.dvi -o lab_dispersion.ps ; ps2pdf lab_dispersion.ps
\documentclass[11pt,titlepage,fleqn]{article}

\usepackage{amsmath}
\usepackage{amssymb}
\usepackage{latexsym}
\usepackage[round]{natbib}
%\usepackage{epsfig}
\usepackage{graphicx}
\usepackage{bm}

\usepackage{url}
\usepackage{color}
%\usepackage{hyperref}

%--------------------------------------------------------------
%       SPACING COMMANDS (Latex Companion, p. 52)
%--------------------------------------------------------------

\usepackage{setspace}    % double-space or single-space
\usepackage{xspace}

\renewcommand{\baselinestretch}{1.2}

\textwidth 460pt
\textheight 690pt
\oddsidemargin 0pt
\evensidemargin 0pt

% see Latex Companion, p. 85
\voffset     -50pt
\topmargin     0pt
\headsep      20pt
\headheight   15pt
\headheight    0pt
\footskip     30pt
\hoffset       0pt

\input{commands_letters}
\input{commands_uaf}
\input{commands_carl}

\newcommand{\blank}{xxxx}

\newcommand{\cyear}{2026}

% provide space for students to write their solutions
\newcommand{\vertgap}{\vspace{1cm}}

\graphicspath{
  {./figures/}
}

\newcommand{\cltag}{GEOS 626/426: Applied Seismology, Carl Tape}
\newcommand{\ptag}{{\bf \textcolor{magenta}{[GEOS 626]}}}
%\newcommand{\ptag}{}

\bibliographystyle{agufull08}

\newcommand{\howmuchtime}{Approximately how much time {\em outside of class and lab time} did you spend on this problem set? Feel free to suggest improvements here.}

%------------------------------------------------------

% modes
\newcommand{\tnl}[2]{\mbox{$_{#1}\ssT_{#2}$}}
\newcommand{\snl}[2]{\mbox{$_{#1}\ssS_{#2}$}}
\newcommand{\snlm}[3]{\mbox{$_{#1}\ssS_{#2}^{#3}$}}
\newcommand{\Tnl}{\mbox{${}_nT_l$}}   % eigenfun
\newcommand{\Wnl}{\mbox{${}_nW_l$}}   % eigenfun
\newcommand{\Ynl}{\mbox{${}_nY_l$}}   % spherical harmonics
% eigenfrequencies
\newcommand{\omnl}{\mbox{${}_{n\hspace{-0.2 mm}}\omega_l$}}
\newcommand{\fnl}{\mbox{${}_{n\hspace{-0.2 mm}}f_l$}}
\newcommand{\omnlm}[1]{\mbox{${}_{n\hspace{-0.2 mm}}\omega_l^{#1}$}}
\newcommand{\fnlm}[1]{\mbox{${}_{n\hspace{-0.2 mm}}f_l^{#1}$}}

\newcommand{\cutoff}[1]{{#1}_{x\bar{n}}}
\newcommand{\Wcol}{\textcolor{blue}{W}}
\newcommand{\Tcol}{\textcolor{red}{T}}

%\newcommand{\rfind}{{\tt fzero}}
\newcommand{\rfind}{{\tt brentq}}
%\newcommand{\sfind}{{\tt ode45}}
\newcommand{\sfind}{{\tt solve\_ivp}}
\newcommand{\maxstep}{{\tt max\_step}}

%------------------------------------------------------


% change the figures to ``Figure L3'', etc
\renewcommand{\thefigure}{L\arabic{figure}}
\renewcommand{\thetable}{L\arabic{table}}
\renewcommand{\theequation}{L\arabic{equation}}
\renewcommand{\thesection}{L\arabic{section}}

\newcommand{\fft}{h}
\newcommand{\ffw}{H}

%--------------------------------------------------------------
\begin{document}
%-------------------------------------------------------------

\begin{spacing}{1.2}
\centering
{\large \bf Lab Exercise: Group speed and phase speed from real seismograms [dispersion]} \\
\cltag\ \\
Last compiled: \today \\
\end{spacing}

%------------------------

\subsection*{Overview}

\begin{itemize}
\item This lab explores dispersion from two seismograms from the same earthquake (\refFig{fig:seis}).

\item The objectives of this lab are:
%
\begin{enumerate}
\item to make frequency-dependent measurements of phase speed and group speed
\item to characterize dispersion from surface wave measurements
\item to relate seismogram-based dispersion analysis to the analytical calculations of dispersion for a layered earth model (\verb+modesA+ and \verb+modesB+)
\end{enumerate}

\item The lab will revisit concepts in
%
\begin{itemize}
\item the Fourier transform notes (\verb+notes_fft.pdf+) and the lab on solar motion (\verb+lab_fft.pdf+)
\item the homeworks on dispersion in a layered earth model (\verb+modesA+ and \verb+modesB+)
\end{itemize}

\item A key point is that in this lab the emphasis is on {\bf phase}, which we use to calculate traveltime differences. In the solar motion lab the emphasis was on {\bf amplitude}, which we used to identify the planets that influenced the sun's motion. (We did use phase to identify which planet was revolving about the sun in a negative sense.)

\item The key Python functions for the lab are \verb+scipy.signal.hilbert()+, \verb+numpy.fft.fft()+, and \verb+numpy.fft.ifft()+. Also required are \verb+scipy.signal.butter()+ and \verb+scipy.signal.filtfilt()+ for bandpass filtering.
%The function \verb+bandpass.m+ is only in the class repository.

\end{itemize}

%------------------------

%\pagebreak
\section{Seismograms}

Using the seismograms in \refFig{fig:seis}, your task is to calculate the group speed $U$ and phase speed $c$ between Pasadena and Needles. These calculations of $U$ and $c$ will depend on period $T$.

\begin{enumerate}
\item Examine the seismograms in \refFig{fig:seis}.
%
\begin{enumerate}
\item Which station is closer to the earthquake? \\
(Assume this is the minor orbit arrival R1.)
\item What is the approximate dominant period in the seismograms?
\item Which station records larger amplitudes?
\end{enumerate}

\item Run the sample program \verb+lab_dispersion.ipynb+ and examine the amplitude spectrum of the two time series. 
%
\begin{enumerate}
\item What is the range of dominant frequencies?
\item What is the corresponding period range?
\item What is the formula for the Nyquist frequency and what is its value here?
\item The Fourier transform of our seismogram can be represented as
%
\begin{equation*}
\ffw(\omega) = \cF[\fft(t)]
\end{equation*}
%
What does the first entry of $\ffw(\omega)$ correspond to? \\
(See \verb+notes_fft.pdf+ or the note \verb+explore these to see...+ in your Python script.)
\item How many entries are in:
\begin{itemize}
\item the time series
\item the Fourier transform of the time series
\item the frequency vector
\end{itemize}
\end{enumerate}

\item Uncomment the plotting commands to plot the envelope functions of the time series. The envelope function is calculated from the Hilbert transform (\verb+scipy.signal.hilbert+). You will need the envelope functions in the next problem.

\end{enumerate}

%-----------------------

\section{Group speed}

\begin{enumerate}
\item Check out the input parameters defined at the start of \verb+lab_dispersion.ipynb+. You will want to use all of these in this problem.

\item Explain the input commands and parameters needed to bandpass filter. Change \verb+False+ to \verb+True+ and see how the randomly generated signal compares with its bandpassed version.

%\item What does the function \verb+bandpass.m+ do?

\item Using \verb+scipy.signal.butter()+ and \verb+scipy.signal.filtfilt()+, compute a bandpass-filtered version of the PAS time series. Given a target period of $f_0 = 1/20$~Hz, use the bandpass frequency limits of $f_a = f_0(1-p)$ and $f_b = f_0(1+p)$ where $p = 0.2$.

\item In a $4 \times 1$ subplot figure, plot the bandpass-filtered versions of PAS for the target periods of 20, 30, 40, and 50~s; plot one time series per subplot. Superimpose the envelope function on each time series.

Repeat for NEE.

\item Using the times associated with the maximum of each bandpass-filtered version, compute the travel time difference between PAS and NEE for each period. Then compute the corresponding group speed dispersion\footnote{This formula assumes that the two stations are on the same great circle path that includes the source, which is the case for us (\refFig{fig:seis} caption).}
%
\begin{equation}
U(T) = \frac{\Delta x}{t_2(T) - t_1(T)},
\label{group}
\end{equation}
%
where $\Delta x = 331$~km. Enter your values in \refTab{tab:group}.

Hint: Use \verb+imax = numpy.argmax(numpy.around(h,4))+ to get the index of the maximum of \verb+h+, then \verb+ti[imax]+ to get the corresponding time.

\item Plot $U$ versus $T$ for the four measurements.
\end{enumerate}

%-----------------------

%\pagebreak
\section{Phase speed}

\begin{enumerate}

\item Phase speed measurements are more difficult than group speed measurements. We need to compute the {\em harmonics} of the time series in \refFig{fig:seis}. A harmonic refers is a signal with a single frequency (``monochromatic'').
You will {\bf not} need to bandpass the signal.
%
\begin{itemize}
\item Determine how the frequency vector in Matlab is stored (recall \verb+lab_fft.pdf+ and \verb+notes_fft.pdf+), then try to compute the harmonics of both time series for the target periods of 20, 30, 40, and 50~s. (See \refFig{fig:spec}.)
\item For each period, make a plot with the harmonic of PAS and the harmonic of NEE, each normalized to 1.
\item Plot using the time limits 2100 to 3100 s (\verb+plt.xlim([2100, 3100])+).
\end{itemize}

\item Experiment with adding two harmonics together. Plot the time series that is the sum of the 20~s and 50~s harmonics for PAS.

\item Then add the harmonic for $f_0$ to the harmonic for $f_0 + \Delta f$, where $\Delta f$ is the increment in the frequency vector. Plot the time series for sum of the 20~s ($f_0$) harmonic and its adjacent harmonic. What is the phenomenon you observe?

Try $f_0$ and $f_0 + 2\Delta f$. How does the modulation frequency depend on the difference between the two harmonic frequencies?

%\pagebreak
\item The phase speed can be estimated from carefully selected points on the harmonics. {\bf It is impossible to identify the ``real'' phase shift between two harmonic signals,} since there are infinite possibilities ($N$ can be any integer). Some prior information is needed to compute the phase shifts. We are ``told'' that realistic values for phase speed in southern California are those listed in \refTab{tab:lims}.

The phase speed can be estimated with the formula (note the similarity with \refeq{group})
%
\begin{equation}
c(T) = \frac{\Delta x}{N T + \Delta t(T)}
\label{phase}
\end{equation}
%
where $N$ is an integer (note: $N \le 0$ is permissible), $\Delta x$ is the distance between stations, and
%
\begin{equation*}
\Delta t(T) = t_2(T) - t_1(T)
\end{equation*}
%
is the phase shift between two time series for a given period $T$, $N$ is an integer (note: $N \le 0$ is permissible). The total travel time of the harmonic wave between the two stations is the sum of (a) an integer number of periods and (b) the phase shift $\Delta t$.

Solve for $\Delta t$ for the case with $N = 0$. Compute the min and max values for $\Delta t$ based on the min and max allowable values of $c$. Also compute $\Delta t_{\rm min}/T$ and $\Delta t_{\rm max}/T$. List the values in \refTab{tab:lims}.

%----------------

\item Plot the unit-normalized harmonics for PAS and NEE for periods 20, 30, 40, and 50~s.

Plot using the time limits 2580 to 2780 s (\verb+plt.xlim([2580, 2780])+).

\item Measure the phase between PAS and NEE for the 20~s harmonic; use the function \verb+markt+ defined for you (note \footnote{See {\tt lab\_fft.ipynb} or {\tt hw\_inv.ipynb} to see how to interact with figures.}) to click points on your plots. 
%
\begin{eqnarray*}
t_1 &=& 
\\
t_2 &=& 
\\
\Delta t &=& t_2 - t_1 =
\end{eqnarray*}
%
Then, using \refEq{phase}, calculate a range of values of $c$ for $N = -5, -4, \ldots, 5$. What value of $N$ gives you a value of $c$ that is within the permissible range? What is your estimated $c$ for this period?

Note: Which time series should your ``reference'' time pick $t_1$ be on?

%----------------

\item Repeat for the other three harmonics, and enter your results in the top half of \refTab{tab:picks}.

%----------------

\item In the bottom half of \refTab{tab:picks}, list the corresponding values of $t_2$ and $\Delta t$  for $N = 0$ and identify the corresponding peaks on the harmonic plots. Note that $t_1$ and $c$ will remain the same.

\item Using your results in \refTab{tab:picks}, plot the phase speed dispersion as $c(T)$ for four points. Also plot the allowable range of $c(T)$ for each period.

\end{enumerate}

%-------------------------------------------------------------

\section{From frequency-dependent surface wave speeds to 3D seismic wave speeds}

From a pair of seismograms we can see how different frequencies of surface waves travel at different speeds between the two stations. In order to oibtain a spatial map of variations in surface wave speeds (group or phase), we need {\bf lots of stations and lots of measurements}. This would allow us to construct spatial maps of wave speed variations for each frequency.

In order to convert the wave speed maps to 3D models of seismic wave speed (\eg $\vs$), we need to know what the sensitivity of each surface wave frequency to some initial (simple) model of wave speed structure. What we need are the eigenfunctions---similar to those that we calculated in the modes and Love wave homeworks.

%-------------------------------------------------------------
%\bibliography{}

%-------------------------------------------------------------

\clearpage\pagebreak

\begin{table}[h]
\centering
\caption[]
{{
Travel time measurements used for calculating group speed (\refeq{group}).
\label{tab:group}
}}
\tgap
\begin{tabular}{|c|c|c|c|c|c|c|}
\hline
$T$, s & $t_1$ & $t_2$ & $t_2 - t_1$ & $U(T)$ \\ \hline
20 &    \hspace{2cm} & \hspace{2cm} & \hspace{1cm} & \hspace{1cm} \\ \hline
30 &   & & & \\ \hline
40 &   & & & \\ \hline
50 &   & & & \\ \hline
\end{tabular}
\end{table}

\begin{table}[h]
\centering
\caption[]
{{
Prior information on limits for allowable phase speeds, which provide limits on the travel time between the two stations. {\bf No measurements are needed for this table.}
\label{tab:lims}
}}
\tgap
\begin{tabular}{|c|c|c|c|c|c|c|}
\hline
$T$, s & $c_{\rm min}$, km/s & $c_{\rm max}$, km/s & $\Delta t_{\rm min}$ ($N=0$) & $\Delta t_{\rm max}$ ($N=0$) & $\Delta t_{\rm min}/T$ & $\Delta t_{\rm max}/T$ \\ \hline
20 & 3.1 & 3.9 & \hspace{2cm} & \hspace{2cm} & \hspace{1cm} & \hspace{1cm} \\ \hline
30 & 3.0 & 4.3 & & & & \\ \hline
40 & 3.3 & 4.5 & & & & \\ \hline
50 & 3.3 & 4.5 & & & & \\ \hline
\end{tabular}
\end{table}

\begin{table}[h]
\centering
\caption[]
{{
Travel time measurements used for calculating phase speed (\refeq{phase}).
\label{tab:picks}
}}
\tgap
\begin{tabular}{|c|c|c|c|c|c|}
\hline
$T$, s & $t_1$, s & $t_2$, s & $\Delta t$, s & $N$ & $c$, km/s \\ \hline\hline
20 & \hspace{2cm} & \hspace{2cm} & \hspace{2cm} & \hspace{2cm} & \hspace{2cm} \\ \hline
30 & & & & & \\ \hline
40 & & & & & \\ \hline
50 & & & & & \\ \hline\hline
& & & & & \\ \hline\hline
%$T$, s & $t_1$, s & $t_2$, s & $\Delta t$, s & $N$ & $c$, km/s \\ \hline
20 & --- & \hspace{2cm} & \hspace{2cm} & 0 & --- \\ \hline
30 & --- & & & 0 & --- \\ \hline
40 & --- & & & 0 & --- \\ \hline
50 & --- & & & 0 & --- \\ \hline
\end{tabular}
\end{table}

%-------------------------------------------------------------

%\clearpage\pagebreak

\begin{figure}[h]
\centering
\includegraphics[width=16cm]{PAS_NEE_seis.eps}
\caption[]
{{
Vertical component seismograms from an earthquake (2002-01-02, \magw{7.3}) in the Vanuatu Islands, southwest Pacific, recorded at two stations in southern California: Pasadena (PAS) and Needles (NEE). These two stations and the epicenter are almost on a great circles, and the distance between the two stations is $\Delta x = 331$~km. The origin time of the event is $t = 0$; therefore the Rayleigh wave arrives at PAS after 2400~s, or 40 minutes.
\label{fig:seis}
}}
\end{figure}

%\begin{figure}
%\begin{center}
%\includegraphics[width=16cm]{PAS_env.eps}
%\caption[]
%{{
%Bandpass filtered versions of the PAS time series in \refFig{fig:seis}.
%\label{fig:PAS_env}
%}}
%\end{center}
%\end{figure}

%\begin{figure}
%\begin{center}
%\includegraphics[width=16cm]{NEE_env.eps}
%\caption[]
%{{
%Bandpass filtered versions of the NEE time series in \refFig{fig:seis}.
%\label{fig:NEE_env}
%}}
%\end{center}
%\end{figure}

\begin{figure}[h]
\centering
\begin{tabular}{cc}
(a) & \includegraphics[width=13.5cm]{PAS_NEE_spec_fneg.eps} \\
(b) & \includegraphics[width=13.5cm]{PAS_NEE_spec.eps}
\end{tabular}
\caption[]
{{
FFT amplitude spectrum of the PAS and NEE seismograms in \refFig{fig:seis}.
Note that this does {\bf not} capture the key quantity of interest, which is the difference in phase between the two seismograms.
(a) Plot using the full range of frequencies.
(b) Plot showing positive frequencies only.
\label{fig:spec}
}}
\end{figure}

%-------------------------------------------------------------
\end{document}
%-------------------------------------------------------------
