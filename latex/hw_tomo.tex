% dvips -t letter hw_tomo.dvi -o hw_tomo.ps ; ps2pdf hw_tomo.ps
\documentclass[11pt,titlepage,fleqn]{article}

\usepackage{amsmath}
\usepackage{amssymb}
\usepackage{latexsym}
\usepackage[round]{natbib}
%\usepackage{epsfig}
\usepackage{graphicx}
\usepackage{bm}

\usepackage{url}
\usepackage{color}
%\usepackage{hyperref}

%--------------------------------------------------------------
%       SPACING COMMANDS (Latex Companion, p. 52)
%--------------------------------------------------------------

\usepackage{setspace}    % double-space or single-space
\usepackage{xspace}

\renewcommand{\baselinestretch}{1.2}

\textwidth 460pt
\textheight 690pt
\oddsidemargin 0pt
\evensidemargin 0pt

% see Latex Companion, p. 85
\voffset     -50pt
\topmargin     0pt
\headsep      20pt
\headheight   15pt
\headheight    0pt
\footskip     30pt
\hoffset       0pt

\input{commands_letters}
\input{commands_uaf}
\input{commands_carl}

\newcommand{\blank}{xxxx}

\newcommand{\cyear}{2026}

% provide space for students to write their solutions
\newcommand{\vertgap}{\vspace{1cm}}

\graphicspath{
  {./figures/}
}

\newcommand{\cltag}{GEOS 626/426: Applied Seismology, Carl Tape}
\newcommand{\ptag}{{\bf \textcolor{magenta}{[GEOS 626]}}}
%\newcommand{\ptag}{}

\bibliographystyle{agufull08}

\newcommand{\howmuchtime}{Approximately how much time {\em outside of class and lab time} did you spend on this problem set? Feel free to suggest improvements here.}

%------------------------------------------------------

% modes
\newcommand{\tnl}[2]{\mbox{$_{#1}\ssT_{#2}$}}
\newcommand{\snl}[2]{\mbox{$_{#1}\ssS_{#2}$}}
\newcommand{\snlm}[3]{\mbox{$_{#1}\ssS_{#2}^{#3}$}}
\newcommand{\Tnl}{\mbox{${}_nT_l$}}   % eigenfun
\newcommand{\Wnl}{\mbox{${}_nW_l$}}   % eigenfun
\newcommand{\Ynl}{\mbox{${}_nY_l$}}   % spherical harmonics
% eigenfrequencies
\newcommand{\omnl}{\mbox{${}_{n\hspace{-0.2 mm}}\omega_l$}}
\newcommand{\fnl}{\mbox{${}_{n\hspace{-0.2 mm}}f_l$}}
\newcommand{\omnlm}[1]{\mbox{${}_{n\hspace{-0.2 mm}}\omega_l^{#1}$}}
\newcommand{\fnlm}[1]{\mbox{${}_{n\hspace{-0.2 mm}}f_l^{#1}$}}

\newcommand{\cutoff}[1]{{#1}_{x\bar{n}}}
\newcommand{\Wcol}{\textcolor{blue}{W}}
\newcommand{\Tcol}{\textcolor{red}{T}}

%\newcommand{\rfind}{{\tt fzero}}
\newcommand{\rfind}{{\tt brentq}}
%\newcommand{\sfind}{{\tt ode45}}
\newcommand{\sfind}{{\tt solve\_ivp}}
\newcommand{\maxstep}{{\tt max\_step}}

%------------------------------------------------------


\usepackage{pifont}  % \ding

\newcommand{\tfiletomo}{{\tt hw\_tomo.ipynb}}
\newcommand{\tfileGik}{{\tt hw\_tomo\_Gik.ipynb}}

%--------------------------------------------------------------
\begin{document}
%-------------------------------------------------------------

\begin{spacing}{1.2}
\centering
{\large \bf Problem Set 9: Seismic tomography using ray theory [tomo]} \\
\cltag\ \\
Assigned: March 30, \cyear\ --- Due: April 6, \cyear\ \\
Last compiled: \today
\end{spacing}

%------------------------

\subsection*{Overview}

\begin{itemize}
%\item WARNING: This document is in a state of transition to reflect the shift from Matlab to Python.

\item The objective of this assignment is to design a synthetic seismic tomography experiment and to familiarize yourselves with the fundamental aspects of a tomographic inversion:
%
\begin{itemize}
\item model parameterization
\item construction of the matrix of partial derivatives (of each measurement with respect to each model parameter)
\item solving the least-squares problem using regularization
\end{itemize}

\item The source--receiver geometry for this problem is shown in \refFig{fig:geometry}.

\item This example problem can be thought of as a surface-wave tomography problem, whereby the unknown model is a spatial map of surface wave phase speed and the observations are measurements of surface wave traveltimes.

The unknown phase speed map may reflect reality or it may be something that is made up.

\item The reading and notes on inverse problems from \verb+hw_inv+ is relevant.
At the end I have listed some additional references.
\nocite{Tape2007,Menke,AsterE2,Tarantola2005}

\item 
%The practical aspects of the assignment are entirely based on Matlab (plus its Mapping Toolbox), and I have provided several scripts that you will need.
% in a reference directory available from the Blackboard website.
The two template scripts are \tfileGik\ and \tfiletomo.
The data files you need are in the directory \verb+data/+. \\

\end{itemize}

%------------------------

\subsection*{Problem 1 (4.0). Constructing the matrix of partial derivatives.}

\begin{enumerate}
\item (0.0) {\em Model parameterization}. A tomographic model is typically described in terms of a perturbation in wave speed structure from some reference model, \ie $\delta\ln c = \delta c/c$.
(Note~\footnote{See the section entitled ``ASIDE: the natural log function used to express tomographic models'' in the math review homework solutions.})
This function may be expanded in terms of a set of $\nparm$ basis functions:
%
\begin{equation}
\delta\ln c(\bx)=\sum_{k=1}^\nparm \delta m_k\,B_k(\bx),
\label{dlnv}
\end{equation}
%
where $\delta m_k$, $k=1,\ldots,\nparm$, represent the model coefficients describing the model perturbation.  Here we choose to use spherical spline basis functions \citep{WangDahlen1995spline,Wang1998}.
\refFig{fig:basis}b shows an example of a spherical spline basis function.

Do the lab exercise in \verb+lab_tomo.pdf+.

%-----------------

\item (1.0) {\em Formulate an entry of the partial derivatives matrix}.
%
%From the class notes and lecture slides,
The ray theory traveltime perturbation for the $i$th source--receiver combination is given by
%
\begin{equation}
\delta T_i = -\int_{\mathrm{ray}_i} c^{-1}\,\delta\ln c\;\rmd s,
\label{ray_tomography}
\end{equation}
%
where $\rmd s$ denotes a segment of the $i$th ray.

\begin{enumerate}
\item (0.8) Assume a homogeneous reference model with $c_0$ everywhere.

Show how \refEq{ray_tomography} leads to 
%
\begin{equation}
\bdelta\bd = \bG\,\bdelta\bem,
\end{equation}
%
where the data vector is given by
%
\begin{equation}
\bdelta\bd = (\delta T_1,\ldots,\delta T_i,\ldots,\delta T_\ndata)^T,
\label{d}
\end{equation}
%
and the model vector is given by
%
\begin{equation}
\bdelta\bem = (\delta m_1,\ldots,\delta m_k,\ldots,\delta m_\nparm)^T,
\label{m}
\end{equation}
%
Write your final equation in matrix-vector schematic form, as done in \verb+lab_linefit+ (line) and \verb+hw_inv+ (ellipse).

\item (0.2) Write the expression for the $G_{ik}$ entry of the partial derivatives matrix~$\bG$.
\end{enumerate}

%-----------------

\item (0.0) Run \tfileGik\ and identify the dimensions of the variables.

%-----------------

\item (3.0) {\em Compute an entry of the partial derivatives matrix}. \\
Modify \tfileGik\ to compute the $\ndata \times \nparm$ (ndata $\times$ nparm) partial derivatives matrix~$\bG$.
Take into account the following details:
%
\begin{itemize}
\item Assume a homogeneous reference phase speed model with $c_0 = 3500$~m/s.

\item Because the reference model is homogeneous, the ray paths are simply great circles.  Use the pre-defined function \verb+geoid_linspace+ to compute the ray paths, and use 1000 segments or so to effectively describe the ray path (\verb+numpts+). The function \verb+arcdist+ will compute the great-circle distance (in degrees) between two points on a sphere.

\item Each measurement involves a single source--receiver pair. Use the source--receiver ordering provided in \tfileGik\ (see also \refFig{fig:geometry}b), which is
%
\begin{verbatim}
i = (isrc-1)*nrec + irec
\end{verbatim}
%
Here index $i$ is the row index (or measurement index; $i = 1,\ldots,\ndata$) for $\bG$, \verb+isrc+ is the source index ($1, \ldots, N_{\rm src}$), and \verb+irec+ is the receiver index ($1, \ldots, N_{\rm rec}$). For example, row $i=3293$ of $\bG$ (and $\bdelta\bd$) corresponds to the pairing of source 25 with receiver 125: $i=(25-1)132 + 125$.

\item Use $\nparm=286$ spherical spline basis functions of order $q=8$; the center-points are in the data file \verb+con_lonlat_q08.dat+.

\item When properly coded, it should take $<$0.1~s to compute a row of $\bG$ and therefore about 3~minutes to compute the entire matrix. Make sure that your code is reasonably fast (though getting the right answer is most important).

\end{itemize}

\begin{enumerate}
\item (0.2) What are the units for $\delta d_i$, $\delta m_k$, and $G_{ik}$?

\item (2.4) Check that the value of $G_{ik}$ with $i=126$ and $k=204$ is $-10.3747$. \\ Show your code for calculating $G_{ik}$ and also your numerical output value.

Note: Based on the indexing above, the $i=126$ measurement corresponds to the ray path between the \verb+isrc=1+ source and the \verb+irec=126+ receiver.

Note: Python is 0-indexed, so the $i$th row in the $\bG$ array will be accessed using $i-1$, and the $k$th column in the $\bG$ array will be accessed using $k-1$.

\item (0.2) What does each row of $\bG$ correspond to?

\item (0.2) What does each column of $\bG$ correspond to?
\end{enumerate}

%-----------------

\item (0.0) Save the partial derivatives matrix
%
\begin{verbatim}
np.save('./Gik',Gik)  % will save the file Gik.mat with the one variable Gik
\end{verbatim}
%
so that you can load it for the next problem.
%
\begin{verbatim}
Gik = np.load('./Gik.npy')   % will load the file Gik.mat with the one variable Gik
\end{verbatim}

\end{enumerate}

%------------------------

%\pagebreak
\subsection*{Problem 2 (4.2). Solving for the unknown wavespeed structure.}

Instructions:
%
\begin{itemize}
\item At this point, you should have saved your partial derivatives matrix $\bG$. \\ Now modify \tfiletomo.

\item In order to do the inverse problem, you will need the data vector of traveltime measurements (\refeq{d}). Load the column vector \verb+measure_vec.dat+, and check that it has dimension \mbox{$3300 \times 1$}.

%\item For this problem, we will make the notation change $\bdelta\bem \rightarrow \bem$ and $\bdelta\bd \rightarrow \dvec$, in order to simplify the notation.
\end{itemize}

\begin{enumerate}
\item (0.5)
\begin{enumerate}
\item (0.1) Write the symbolic expression for the solution $\bem$ to the least squares problem $\bG\bdelta\bem=\bdelta\bd$ in the case where $\bG^T\bG$ is full rank.

\item (0.4) List the dimensions and units of $\bdelta\bem$, $\bdelta\bd$, $\bG$, $\bG^T\bG$, and $\bG^T\bdelta\bd$.
\end{enumerate}

%------------

\item (0.3) Compute $\bdelta\bem$ using your formula.  What do you get, and why is this the case?

%------------

\item (1.0) If $\bG^T\bG$ is not full rank, then a damping parameter $\lambda$ may be introduced to stabilize (or ``regularize'') the inversion.  One possible solution to the problem is:
%
\begin{equation}
\bdelta\bem_{\lambda} = \left(\bG^T\bG+\lambda^2\bI\right)^{-1}\bG^T\bdelta\bd.
\end{equation}
%
\begin{enumerate}

\item (0.6) Compute $\bdelta\bem_{\lambda}$ for $\lambda = 10$. \\
List the sum of the squares of the entries of the residual vector $\br_\lambda = \bG\bdelta\,\bem_\lambda-\bdelta\bd$; this is the {\em misfit norm} (squared).
List the sum of the squares of the entries of the model vector $\bdelta\bem_{\lambda}$; this is the {\em model norm} (squared).

%------------

\item (0.0) Compute $\bdelta\bem_{\lambda}$ for a range of $\lambda$ values: for example, try

\verb+lamvec = 10**np.linspace(np.log10(minlam),np.log10(maxlam),num=numlam)+
%\verb+lamvec = 10.^linspace(log10(minlam), log10(maxlam), numlam);+

with $\lambda_{\rm min} = 0.01$ and $\lambda_{\rm max} = 1000$ (specify a value for \verb+numlam+).

%------------

\item (0.4) Plot the misfit norm ($y$-axis) versus model norm ($x$-axis) using log-scaled values.

%The misfit norm is the sum of the squares of the residuals, where the $i$the residual is $(\bG\bem-\dvec)_i$. The model norm is the sum of the squares of the model parameters $m_k$.

I recommend transforming the quantities by $\log_{10}$, then using the \verb+plot+ command, rather than dealing with the \verb+plt.loglog()+ command.

\end{enumerate}

%------------

\item (1.0) 

\begin{enumerate}
\item (0.6) Plot some of these model vectors by expanding them in the spherical spline basis functions. Use the same color scale for each plot: \verb+vmin=-0.05,vmax=0.05+ parameters in \verb+plt.pcolormesh()+. (See~Note\footnote{If you want the {\tt seis} color scale from GMT, then set {\tt cmap=seis\_cmap}.}.) {\bf Superimpose the source locations in each plot.}

\item (0.4) Comment on any trends that you see in phase speed maps that you produce. \\
What happens as $\lambda \rightarrow \infty$?
\end{enumerate}

Notes:
%
\begin{itemize}
\item I have pre-computed the necessary matrix $\bB$ in \tfiletomo, such that the function value $\delta\ln c(\phi_j, \theta_j)$ is computed by the dot product $\bB(j,:) \cdot \bdelta\bem$. This operation is represented by \refEq{dlnv} and can be thought of as
%
\begin{equation}
\bdelta\bc_{\lambda} = \bB\,\bdelta\bem_{\lambda},
\label{Bexpand}
\end{equation}
%
where $\bdelta\bc_{\lambda} = \delta\ln c(\phi, \theta)$ is a vector of phase speed perturbations.

\item The simplest way to plot $\bc_{\lambda}$ is probably with the \verb+plt.scatter()+ command. \\ But using something like \verb+plt.pcolormesh()+ will produce prettier results.

\end{itemize}

%\pagebreak
\item (1.0) \ptag\
%
\begin{enumerate}
\item (0.7) Perform an inversion using only the first five sources; this requires using a reduced form of your original partial derivatives matrix. Use the same damping parameters as before. Include plots as before, and discuss your results.

Note: Plot only the five sources that were used in the inversion.

\item (0.3) Repeat using only the first source.
\end{enumerate}

\item (0.4) What are two ways to stabilize the inverse problem based on {\bf the experimental design}? Hint: How can we construct a matrix $\bG^T\bG$ with fewer zeros?

\end{enumerate}

% %------------------------

\pagebreak
\subsection*{Problem 3 (1.8). Connections with real tomography problems.}

For your answers to this problem, try to imagine explaining the concepts to a fellow graduate student who is not a seismologist. (This should make your answers more explicit.)

\begin{enumerate}
\item (0.6) Our data vector contains a set of traveltime differences between a wave in our model and a wave in the target model (\ie the unknown Earth model). For problems using recorded seismic data, how is this measurement made?

\item (0.6) After the first iteration, we no longer have a uniform ``reference'' model for phase speed. How will we need to adapt our algorithm to accommodate iterations of the phase speed model?

Hint: How will $\bG$, $\dvec$, and $\bem$ change?

\item (0.6) In a real problem we might make measurements at several different periods. How can the phase speed for different periods be used to obtain a 3D model of shear wave structure?

Hint: Think about the modes homework.

\end{enumerate}

%------------------------

\subsection*{Problem} \howmuchtime\

%-------------------------------------------------------------
%\pagebreak
\bibliography{uaf_abbrev,uaf_main,uaf_carletal}
%-------------------------------------------------------------

\clearpage\pagebreak

%\vspace{1cm}
\begin{figure}
\hspace{-0.5cm}
\includegraphics[width=17cm]{socal_03b.eps}
\caption[Source--receiver geometry for southern California]
{{
{\em Left}: Example phase speed map and faults for Southern California. The phase speed map is for Rayleigh waves with periods of 20~seconds.
{\em Right}: Source--receiver geometry for the problem in this homework. The \ding{73} symbols denote the locations of 25 earthquakes; the $\circ$ symbols denote the locations of 132 broadband receivers in the Southern California Seismic Network.
\label{fig:geometry}
}}
\end{figure}

\begin{figure}[p]
\hspace{-1.5cm}
\includegraphics[width=19cm]{talk_Gik03_5_ray_hw.eps}
\caption[Basis function]
{{
Example computation for an entry, $G_{ik}$, of the partial derivatives matrix $\bG$, using rays. The row index $i$ is the source-receiver combination, the column index $k$ is the basis function index. The source is denoted by the \ding{73}, the receiver is denoted by the $\triangle$, and the $\circ$ shows the center-point of the spherical spline in (b).
(a)~Ray path for source number~1 and receiver number~126, corresponding to the $i=126$ index of the $\ndata=3300$ ray paths.
(b)~$B_{203}(\bx)$, the spherical spline basis function for index $k=203$. Also shown are the center-points of the $\nparm=286$ spherical splines.
(c)~Spline $B_{203}$ evaluated along the ray path. The value of the phase speed for the reference model is constant, so $G_{ik} = (-1/c) \int_{{\rm ray}_i} B_k\,\rmd s$. In this example, $G_{ik} = -1/(3500\,{\rm m\,s^{-1}})\;(2.45 \times 10^4\,{\rm m})  = -6.98$~s. 
\label{fig:basis}
}}
\end{figure}

%-------------------------------------------------------------
\end{document}
%-------------------------------------------------------------
