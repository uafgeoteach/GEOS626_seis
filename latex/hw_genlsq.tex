% HOMEWORK ASSIGNMENT IN GEOS 626 [hw_genlsq]
% LAB IN GEOS 627 [lab_iter]
%
% dvips -t letter hw_genlsq.dvi -o hw_genlsq.ps ; ps2pdf hw_genlsq.ps
% dvips -t letter lab_iter.dvi -o lab_iter.ps ; ps2pdf lab_iter.ps
%
% THE POINT TOTALS ARE IRRELEVANT FOR THE GEOS 627 LAB EXERCISE.
%
% COPIED AND ADAPTED FROM appliedseis/homework/hw_genlsq.tex
% search for 627 to find some of the changes made for 627
%
\documentclass[11pt,titlepage,fleqn]{article}

\usepackage{amsmath}
\usepackage{amssymb}
\usepackage{latexsym}
\usepackage[round]{natbib}
%\usepackage{epsfig}
\usepackage{graphicx}
\usepackage{bm}

\usepackage{url}
\usepackage{color}
%\usepackage{hyperref}

%--------------------------------------------------------------
%       SPACING COMMANDS (Latex Companion, p. 52)
%--------------------------------------------------------------

\usepackage{setspace}    % double-space or single-space
\usepackage{xspace}

\renewcommand{\baselinestretch}{1.2}

\textwidth 460pt
\textheight 690pt
\oddsidemargin 0pt
\evensidemargin 0pt

% see Latex Companion, p. 85
\voffset     -50pt
\topmargin     0pt
\headsep      20pt
\headheight   15pt
\headheight    0pt
\footskip     30pt
\hoffset       0pt

\input{commands_letters}
\input{commands_uaf}
\input{commands_carl}

\newcommand{\blank}{xxxx}

\newcommand{\cyear}{2026}

% provide space for students to write their solutions
\newcommand{\vertgap}{\vspace{1cm}}

\graphicspath{
  {./figures/}
}

\newcommand{\cltag}{GEOS 626/426: Applied Seismology, Carl Tape}
\newcommand{\ptag}{{\bf \textcolor{magenta}{[GEOS 626]}}}
%\newcommand{\ptag}{}

\bibliographystyle{agufull08}

\newcommand{\howmuchtime}{Approximately how much time {\em outside of class and lab time} did you spend on this problem set? Feel free to suggest improvements here.}

%------------------------------------------------------

% modes
\newcommand{\tnl}[2]{\mbox{$_{#1}\ssT_{#2}$}}
\newcommand{\snl}[2]{\mbox{$_{#1}\ssS_{#2}$}}
\newcommand{\snlm}[3]{\mbox{$_{#1}\ssS_{#2}^{#3}$}}
\newcommand{\Tnl}{\mbox{${}_nT_l$}}   % eigenfun
\newcommand{\Wnl}{\mbox{${}_nW_l$}}   % eigenfun
\newcommand{\Ynl}{\mbox{${}_nY_l$}}   % spherical harmonics
% eigenfrequencies
\newcommand{\omnl}{\mbox{${}_{n\hspace{-0.2 mm}}\omega_l$}}
\newcommand{\fnl}{\mbox{${}_{n\hspace{-0.2 mm}}f_l$}}
\newcommand{\omnlm}[1]{\mbox{${}_{n\hspace{-0.2 mm}}\omega_l^{#1}$}}
\newcommand{\fnlm}[1]{\mbox{${}_{n\hspace{-0.2 mm}}f_l^{#1}$}}

\newcommand{\cutoff}[1]{{#1}_{x\bar{n}}}
\newcommand{\Wcol}{\textcolor{blue}{W}}
\newcommand{\Tcol}{\textcolor{red}{T}}

%\newcommand{\rfind}{{\tt fzero}}
\newcommand{\rfind}{{\tt brentq}}
%\newcommand{\sfind}{{\tt ode45}}
\newcommand{\sfind}{{\tt solve\_ivp}}
\newcommand{\maxstep}{{\tt max\_step}}

%------------------------------------------------------

%\input{hw627_header}

%\newcommand{\tfile}{{\tt lab\_iter.ipynb}}
\newcommand{\tfile}{{\tt hw\_genlsq.ipynb}}
\newcommand{\tfileFE}{{\tt forward\_epicenter.ipynb}}
\newcommand{\tfileOM}{{\tt optimization\_method.ipynb}}

\renewcommand{\baselinestretch}{1.1}

\graphicspath{
  {./figures/}
  {./seis/figures/}    % only needed when running from inv
}

%--------------------------------------------------------------
\begin{document}
%-------------------------------------------------------------

\begin{spacing}{1.2}
\centering
{\large \bf Problem Set 10: Earthquake location with generalized least squares [genlsq]} \\
%{\large \bf Lab Exercise: Earthquake location with generalized least squares [iter]} \\
\cltag\ \\
Assigned: April 13, \cyear\ --- Due: April 25, \cyear\ \\
Last compiled: \today
\end{spacing}

%------------------------

\vspace{1cm}
\begin{figure}[h]
\centering
\includegraphics[width=13cm]{optimhw_fig02.eps}
\caption[Source-receiver geometry]
{{
Source--receiver geometry for the earthquake location problem. The ray paths are drawn between the (fixed) receivers and the initial epicenter $\bem_{\rm initial}$.
The target data are computed using the target model \textcolor{red}{$\bem_{\rm target}$}.
The data $\dvec$ are constructed by adding errors to the target data.
Both $\bem_{\rm initial}$ and $\bem_{\rm target}$ are samples within the prior distribution, which is centered at \textcolor{blue}{$\bem_{\rm prior}$} and represented by the 1000 dots. Alternatively, one could choose \textcolor{blue}{$\bem_{\rm prior}$} to be $\bem_{\rm initial}$.
\label{fig:srcrec}
}}
\end{figure}

%------------------------

%\vspace{-1cm}
\pagebreak
\subsection*{Overview and instructions}

This is a follow-up to the simple iterative Newton method example in \verb+lab_newton.pdf+. Be sure that you have done that exercise before beginning here.

Generalized least squares was introduced by Tarantola \citep{TarantolaValette1982quest,TarantolaValette1982nonlinear}; a comprehensive review can be found in \citet{Tarantola2005}, available as a pdf (note~\footnote{The author's website states: ``If you can not purchase the book, download it for free.''}). Generalized least squares is a Bayesian probabilistic inverse problem that assumes Gaussian uncertainties for model parameters and data.

Following \citet[][p.~205]{Tarantola2005}, the general form of a {\bf nonlinear least-squares misfit function} is
%\footnote{Here I am using $\cprior$ whereas Tarantola uses $\covm$.}
%
\begin{eqnarray}
\misfitvar(\mvec) &=& \misfitvar_{\rm D}(\mvec) + \misfitvar_{\rm M}(\mvec) 
\\
&=& \hlf \left\| \bg(\mvec) - \dvec \right\|^2_{\rm D} + \hlf \left\| \mvec - \mprior \right\|^2_{\rm M}
\\
&=& \hlf \left( \bg(\mvec) - \dvec\right)^T \; \covdi \left( \bg(\mvec) - \dvec \right)
\;+\; \hlf \left( \mvec - \mprior \right)^T \; \cpriori \left( \mvec - \mprior \right).
\nonumber \\
%\label{misfit_gen}
\end{eqnarray}
%
Note these differences in notation (see \refTab{tab:vars}):

\vspace{0.2cm}
\begin{tabular}{l|c|c|c}
\hline
                                    & Tarantola         & Aster    & our \\ 
variable                            & notation (p. 66) & notation (p. 286)  & notation \\ \hline
prior model covariance matrix       & $\bC_{\rm M}$                & $\bC_{\rm M}$    & $\cprior$    \\ \hline
posterior model covariance matrix   & $\widetilde{\bC}_{\rm M}$    & $\bC_{M'}$      & $\cpost$     \\ \hline
prior model vector                  & $\mprior$                  & $\mprior$      & $\mprior$     \\ \hline
posterior model vector              & $\widetilde{\bem}$         & $\mvec_{\rm MAP}$ & $\mpost$     \\ \hline
\hline
\end{tabular}
\vspace{0.2cm}

\nocite{Aster}

\medskip\noindent
A Python solution to a least squares problem may be as simple as \verb+numpy.linalg.solve(G,d)+ or \verb+numpy.linalg.pinv(G)@d+, but what is the meaning (and computation) behind these expressions? The objective of this assignment is to demystify some of the black-box elements of least-squares inverse problems. This will involve a combination of reading \citet{Tarantola2005}, solving some equations, and implementing the equations within Jupyter Notebook. We approach these topics with the classic example of earthquake location.

% comment out for 627:
See \citet[][Chapter 7]{SteinWysession} for an introduction to the earthquake location problem. However, note that they do not approach the problem from a generalized least squares perspective, as we do in this assignment. In particular, they do not consider the model misfit term $(\mvec-\mprior)^T \cpriori (\mvec-\mprior)$. But they do introduce the concept of the data covariance matrix and posterior model covariance matrix on p. 421.

Throughout the assignment, and in the code, keep in mind the interchangability of these terms:
%
\begin{eqnarray*}
\mpost &=& \mvec_{\rm final}
\\
\mtarget &=& \mvec_{\rm true}
\end{eqnarray*}
%
The posterior model is the same as the final model of an iterative problem. And the ``target model'' is the same as the ``true model'': this is the model used to generate the ``target data'' or ``true data''. We will use the notation $\mpost$ and $\mtarget$.

%------------------------

% delete this problem for 627 lab
\pagebreak
\subsection*{Problem 0 (2.0). \ptag\ Generalized least squares solutions}

Here you will derive formulas for generalized least squares under the assumption of a linear model $\bg(\mvec) = \bG\mvec$.
%
\begin{enumerate}
\item (1.5) Show the equivalence of the three expressions for the posterior Gaussian that are listed in \citet[][p.~66]{Tarantola2005}:
%
\begin{eqnarray}
\mpost &=& \left(\bG^{\tssT} \covdi \bG + \cpriori \right)^{-1}
\left( \bG^{\tssT} \covdi \bd + \cpriori\mprior \right)
\label{mpost1}
\\
&=& \mprior + \left(\bG^{\tssT} \covdi \bG + \cpriori \right)^{-1} \bG^{\tssT} \covdi \left( \bd - \bG \mprior\right)
\label{mpost2}
\\
&=& \mprior + \cprior \bG^{\tssT} \left( \bG \cprior \bG^{\tssT} + \Cd\right)^{-1} \left( \bd - \bG \mprior\right)
\label{mpost3}
\end{eqnarray}
%
HINT: You may find it useful to use the two matrix identities in \citet[][Eq.~6.525]{Tarantola2005} (which are serived in \verb+notes_tarantola.pdf+), as well as the expressions for the posterior covariance,
%
\begin{eqnarray}
\cpost &=& \left(\bG^{\tssT} \covdi \bG + \cpriori \right)^{-1}
\label{cpost1}
\\
&=& \cprior - \cprior \bG^{\tssT} \left( \bG \cprior \bG^{\tssT} + \covd \right)^{-1} \bG \cprior
\label{cpost2}
\end{eqnarray}

\item (0.5) List each matrix in \refEqab{mpost1}{cpost2} that must be inverted. Also list the dimensions of each matrix, assuming that there are $\nparm$ model parameters and $\ndata$ observations.
\end{enumerate}

%------------------------

%\pagebreak
\subsection*{Problem 1 (2.0). Earthquake location: theory}

\begin{enumerate}
\item (0.5) Derive an equation for the arrival time $t_r$ of a seismic wave that originates at source location ($x_s, y_s$) at origin time $t_s$ and is recorded at receiver location ($x_r, y_r$). Assume that the medium is homogeneous and characterized with a single wavespeed $V$.

\item (0.3) Let $t_r = g(\mvec)$, where $\mvec = (x_s,y_s,t_s,V)$.
Is $g$ a linear function of $t_s$? Is it a linear function of $x_s$, $y_s$, or $V$?
Is it a linear function of $\mvec$?

\item (0.2) Apply the variable transformation $v = \ln(V/V_0)$, where $V_0$ is a normalizing constant such that $v$ is unitless. Write $g(\mvec)$ in terms of the model parameters $x_s$, $y_s$, $t_s$, and $v$.

\item (0.8) Write expressions for the partial derivatives of $g(\mvec)$: $\partial g/ \partial x_s$, $\partial g/ \partial y_s$, etc.

\item (0.0) What is the dimension of the matrix of partial derivatives $\bG$, represented by
%
\begin{equation}
G_{ik} = \partial g_i/ \partial m_k
\end{equation}
%
where $i$ is the receiver (or arrival time) index. (Here the subscript $i$ for data index and $r$ for receiver index are the same, since there is one measurement per receiver.)

\item (0.2) Write the matrix-schematic form for $G_{ik}$. Do not write out the full expressions.
For example, the upper left entry of your matrix should be $\partial g_1/\partial x_s$.
Use $\cdots$ to indicate an entry within a sequence.

\end{enumerate}

%------------------------

%\pagebreak
\subsection*{Problem 2 (1.0). Earthquake location: understanding the code}

\begin{itemize}
\item Be sure you have done Problem 1 before starting this problem.

\item I have provided a set of Python notebooks to get you started on this problem. {\bf Do not change any of the user parameters until you have finished these questions. The questions depend on the default settings, as given.}

\item Note: Generating samples (from $\cprior$ and $\covd$) is not required to solve this problem. We use the samples mainly to ``illuminate'' the covariance matrices.

\item It may help to relate this problem to a real-world example of earthquake location. Imagine that you detect a P wave at some particular station (``station X''). The next closest station, which has not yet ``seen'' the P wave, is 20~km away. At that moment you might assume a prior model for the epicenter $(x_s, y_s)$ that is centered on station X and has a Gaussian distribution characterized by $\sigma = 20$~km. 
Your {\bf prior model} for the structure of the medium is assumed to be homogeneous with P-velocity reference value $V = 5$~km/s (with some associated uncertainty). Your prior model for the origin time could be the arrival time at station X (with some associated uncertainty).

You then record the P wave arrival times at all other stations. You choose an {\bf initial model}; a sensible choice might be the prior model (see above). With this initial model, you compute predicted arrival times at all stations. Using the generalized least-squares misfit function (see Overview and instructions), you quantify the misfit between your predicted arrival times and the observed arrival times. The rest is optimization.

\end{itemize}

%----------------------------------------------

%\clearpage\pagebreak
\begin{enumerate}
\item (0.5)
Run \tfile.
%\footnote{You do not need to enter any value at the prompt. If you hit ``0'' and hit return, then it will exit. If you hit any non-zero number and hit return, you will see an error message.}
You should generate six figures, one of which is \refFig{fig:srcrec}. Then the code will crash because some code still needs to be written.

Examine \tfile\ and \tfileFE\ to see what is being plotted in these figures. Answer the following questions ``in words,'' that is, without excerpts from the code.

\begin{enumerate}
\item (0.0) In the code, how is the prior model and the prior model covariance matrix determined?

\vspace{1cm}

\item (0.1) Explain how the initial model and target model are determined (in the hypothetical case that \verb+irandom_initial_model=1+ and \verb+irandom_target_model=1+.)

\begin{itemize}
\item initial model
\item target model
\end{itemize}

\item (0.2) Explain what the ``observations'' are and how they are generated.

\vspace{1cm}

\item (0.2) What do the functions \verb+tt+, \verb+d+, \verb+G+, and \verb+S+ do? Put your responses in \refTab{tab:funs}.

\item (0.0) What are the two main reasons for generating samples of $\cprior$?

\vspace{1cm}

\end{enumerate}

%------------------


\item (0.4) 

\begin{enumerate}
\item (0.0) Write down the quasi-Newton algorithm from \citet[][Eq.~6.319]{Tarantola2005}\footnote{Tarantola calls this the Newton aglorithm, but right before this, he states that he's using ``the (usual approximation of the) Hessian'' that does not include second derivatives of the forward model. This makes it the quasi-Newton algorithm. Also known as the Gauss-Newton algorithm.}. Use $\nu_n = 1$, use $\cprior$ instead of $\covm$, and use $\bd_n = \bg(\mvec_n)$ to represent the predictions of the forward model.

\vspace{1cm}

\item (0.1) Complete \refTab{tab:vars}.

\item (0.1) Compare Eq.~6.319 with Eq.~6.291,
%
\begin{equation}
\mvec_{n+1} \approx \mvec_{n} - \bHh^{-1}_n \bgammah_n.
\label{mn}
\end{equation}
%
What are the equations for $\bgammah$ and $\bHh$?

\vspace{1cm}

\item (0.1) Rewrite \refEq{mn} in terms of $\bgamma$ and $\bH$ instead of $\bgammah$ and $\bHh$. \\ (See \verb+notes_tarantola.pdf+)

\vspace{1cm}

\item (0.1) List full expressions for $\bH$ and $\bgamma$ (no hat-notation should appear). These are what you will be coding in the lab.

\vspace{1cm}

\item (0.1) From Section 3.2.3, what is the equation for $\cpost$?

\vspace{1cm}

\end{enumerate}

\end{enumerate}

%------------------

%\clearpage\pagebreak
\subsection*{Problem 3 (2.5). Earthquake location: adapting the code}

\begin{enumerate}

\item (0.0) Try the following tests to better understand \refFig{fig:srcrec}. \\
{\bf Be sure to undo your changes when you are done.}
%
\begin{enumerate}
\item In \tfileFE, change \verb+run_forward_epicenter+ to \verb+True+, then run the notebook a couple times. \\
What dots remain fixed? What dots change?

\item Change $\sigma_{x_s}$ from 10~km to 20~km, then run \tfileFE. What do you see?

\item Instead of generating samples of the prior model using \verb+np.random.randn+, see what happens when you use \verb+np.random.rand+. Why is this the case?
\end{enumerate}

\item (1.0)
%
\begin{enumerate}
\item (0.5) In the space provided in \tfile\ (or \tfileOM, if present), implement the quasi-Newton algorithm using $\bH$ and $\bgamma$ (rather than using $\bHh$ and $\bgammah$).
% comment out for 627:
Show your lines of code.

% comment out for 627:
\item (0.5) Include the automatically-generated plot of convergence curves.

\item (0.0) Make sure you understand the text output that is generated during the iterative convergence.
\end{enumerate}

%------------------

\item (1.0) In the space provided in \tfile, compute the:
%
\begin{itemize}
\item posterior model
\item posterior covariance matrix
\item arrival time predictions
\end{itemize}

% comment out for 627:
Show your lines of code.

List $\mpost$ and $\cpost$ after \textcolor{red}{eight iterations}. (The initial model is $\mvec_0$, the first iteration is $\mvec_1$.) Note that the posterior model is the final model of your iterative algorithm. Check that:
%
\begin{eqnarray}
\mpost = \bem_{8} =
\left[
\begin{array}{rl}
20.7327 & {\rm km} \\
45.7992 & {\rm km} \\
15.6755 & {\rm s} \\
1.9781  &  \\
\end{array}
\right]
\end{eqnarray}

%------------------

\item (0.5) 
\begin{enumerate}
\item (0.1) In the space provided in \tfile, use $\cpost$ to compute the posterior correlation matrix. Show your lines of code, and display the correlation matrix. A Python function equivalent to the Matlab command \verb+corrcov+ is provided. Do not use the function \verb+corrcov+ in your primary computation, but you can use this command to check.

\item (0.3) List the posterior standard deviations for each model parameter. Include units.

\item (0.1) What is the correlation matrix for $\cprior$?
\end{enumerate}

\end{enumerate}

%------------------

%\pagebreak
\subsection*{Problem 4 (2.5). Earthquake location: interpreting the results}
% 627
%\subsection*{[OPTIONAL] Problem 4 (2.7). Earthquake location: interpreting the results}

\begin{enumerate}

\item (0.2)
\begin{enumerate}
\item (0.1) List numbers to show how the uncertainty estimates for the model parameters change from the prior model to the posterior model.
\item (0.1) List the parameters in order of greatest reduction in uncertainty to least reduction in uncertainty.
\end{enumerate}

%------------------

\item (0.3) Does $\mvec_{\rm post}$ match $\mvec_{\rm target}$? Why or why not?

%------------------

\item (0.3) Express the solution $\mvec_{\rm post}$ with units. List all numbers to 0.01 precision.

%------------------

\item (0.5) What is the {\em physical meaning} of a {\bf positive correlation entry} in the model correlation matrix?
\begin{enumerate}
\item (0.1) List the largest positive off-diagonal correlation coefficient.
\item (0.4) Infer why this is present.
\end{enumerate}

%------------------

\item (0.5) What is the {\em physical meaning} of a {\bf negative correlation entry} in the model correlation matrix?
\begin{enumerate}
\item (0.1) List the largest negative off-diagonal correlation coefficient.
\item (0.4) Infer why this is present.
\end{enumerate}

%------------------

\item (0.2)
%
\begin{enumerate}
\item (0.1) How many iterations are needed for convergence in this example? \\
      What indications do we have that convergence has occurred?
\item (0.1) Explain the relationships among the convergence curves for the data misfit, model misfit, and total misfit.
\end{enumerate}

%------------------

% comment out for 627:
%\iffalse
\item (0.5) Set \verb+irandom_initial_model = 1+, \verb+irandom_target_model = 1+, and \verb+idata_errors = 2+. Then re-run.
%
\begin{enumerate}
\item (0.1) Include two figures: (1) convergence curves, (2) posterior distribution plotted on \refFig{fig:srcrec} but without ray paths. \\
{\bf Note: these figures are automatically generated by \tfile.}
\item (0.1) How many iterations are needed for convergence?
\item (0.2) Try a few runs. \\
Does the positive correlation that you previously identified continue to appear? \\
Does the negative correlation that you previously identified continue to appear?
\item (0.1) Is there any pattern to the position of $\mpost$ relative to $\mtarget$?
\end{enumerate}
%\fi

\end{enumerate}

%------------------------

%\pagebreak

% comment out for 627
\subsection*{Problem} \howmuchtime\

%-------------------------------------------------------------
\bibliography{uaf_abbrev,uaf_main,uaf_source,uaf_carletal,uaf_alaska}

%------------------

\clearpage\pagebreak
\input{hw_genlsq_table}

%-------------------------------------------------------------
\end{document}
%-------------------------------------------------------------
