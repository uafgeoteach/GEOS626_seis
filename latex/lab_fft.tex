% dvips -t letter lab_fft.dvi -o lab_fft.ps ; ps2pdf lab_fft.ps
\documentclass[11pt,titlepage,fleqn]{article}

\usepackage{amsmath}
\usepackage{amssymb}
\usepackage{latexsym}
\usepackage[round]{natbib}
%\usepackage{epsfig}
\usepackage{graphicx}
\usepackage{bm}

\usepackage{url}
\usepackage{color}
%\usepackage{hyperref}

%--------------------------------------------------------------
%       SPACING COMMANDS (Latex Companion, p. 52)
%--------------------------------------------------------------

\usepackage{setspace}    % double-space or single-space
\usepackage{xspace}

\renewcommand{\baselinestretch}{1.2}

\textwidth 460pt
\textheight 690pt
\oddsidemargin 0pt
\evensidemargin 0pt

% see Latex Companion, p. 85
\voffset     -50pt
\topmargin     0pt
\headsep      20pt
\headheight   15pt
\headheight    0pt
\footskip     30pt
\hoffset       0pt

\input{commands_letters}
\input{commands_uaf}
\input{commands_carl}

\newcommand{\blank}{xxxx}

\newcommand{\cyear}{2026}

% provide space for students to write their solutions
\newcommand{\vertgap}{\vspace{1cm}}

\graphicspath{
  {./figures/}
}

\newcommand{\cltag}{GEOS 626/426: Applied Seismology, Carl Tape}
\newcommand{\ptag}{{\bf \textcolor{magenta}{[GEOS 626]}}}
%\newcommand{\ptag}{}

\bibliographystyle{agufull08}

\newcommand{\howmuchtime}{Approximately how much time {\em outside of class and lab time} did you spend on this problem set? Feel free to suggest improvements here.}

%------------------------------------------------------

% modes
\newcommand{\tnl}[2]{\mbox{$_{#1}\ssT_{#2}$}}
\newcommand{\snl}[2]{\mbox{$_{#1}\ssS_{#2}$}}
\newcommand{\snlm}[3]{\mbox{$_{#1}\ssS_{#2}^{#3}$}}
\newcommand{\Tnl}{\mbox{${}_nT_l$}}   % eigenfun
\newcommand{\Wnl}{\mbox{${}_nW_l$}}   % eigenfun
\newcommand{\Ynl}{\mbox{${}_nY_l$}}   % spherical harmonics
% eigenfrequencies
\newcommand{\omnl}{\mbox{${}_{n\hspace{-0.2 mm}}\omega_l$}}
\newcommand{\fnl}{\mbox{${}_{n\hspace{-0.2 mm}}f_l$}}
\newcommand{\omnlm}[1]{\mbox{${}_{n\hspace{-0.2 mm}}\omega_l^{#1}$}}
\newcommand{\fnlm}[1]{\mbox{${}_{n\hspace{-0.2 mm}}f_l^{#1}$}}

\newcommand{\cutoff}[1]{{#1}_{x\bar{n}}}
\newcommand{\Wcol}{\textcolor{blue}{W}}
\newcommand{\Tcol}{\textcolor{red}{T}}

%\newcommand{\rfind}{{\tt fzero}}
\newcommand{\rfind}{{\tt brentq}}
%\newcommand{\sfind}{{\tt ode45}}
\newcommand{\sfind}{{\tt solve\_ivp}}
\newcommand{\maxstep}{{\tt max\_step}}

%------------------------------------------------------


% change the figures to ``Figure L3'', etc
\renewcommand{\thefigure}{L\arabic{figure}}
\renewcommand{\thetable}{L\arabic{table}}
\renewcommand{\theequation}{L\arabic{equation}}
\renewcommand{\thesection}{L\arabic{section}}

\renewcommand{\baselinestretch}{1.0}

%--------------------------------------------------------------
\begin{document}
%-------------------------------------------------------------

\begin{spacing}{1.2}
\centering
{\large \bf Lab Exercise: Discrete Fourier Transform of solar motion [fft]} \\
\cltag\ \\
Last compiled: \today \\
\end{spacing}

%------------------------

\section*{Instructions}

The data set \verb+solarshort.dat+ is a 1024-point synthetic time series of the motion of our Sun due to planet rotations. The three columns are time in Earth years, $x$ position of the Sun in astronomical units (AU), and $y$ position of the Sun. In this hypothetical problem, {\bf one or more of the planets is revolving in an opposite sense from the others.} Your task is to identify the planets in the frequency spectrum, while getting some appreciation for the meaning of the Discrete Fourier Transform with regard to negative frequencies. For some basic information on planets, see these websites:
%
\begin{verbatim}
http://ssd.jpl.nasa.gov/?planet_phys_par
http://ssd.jpl.nasa.gov/?constants
\end{verbatim}

\begin{figure}[h]
\centering
\includegraphics[width=16cm]{fft_solar_data.eps}
\caption[]
{{
Solar motion due to planetary influences. Upper left represents a $\sim$100-year record; upper right is a $\sim$400-year record.
\label{fig:solar}
}}
\end{figure}

%------------------------

\begin{spacing}{2.0}

\section*{Questions}

\begin{enumerate}

\item Run \verb+lab_fft.ipynb+ to generate the plot in \refFig{fig:solar}.

Examine the (synthetic) time series of solar motion and the text output in the command window.
%
\begin{enumerate}
\item What is the main period of oscillation?
\item What planet is responsible for the dominant period?
\item Write the solar motion $c(t)$ in terms of $x(t)$ and $y(t)$. (Examine the code.)
\item What is the number of points $n$ in the discretized version of $c(t)$? 
\item What does the function \verb+fftvec()+ do?
\item What does the function \verb+np.fft.fft+ do?
\end{enumerate}

%---------

\item For the {\bf shorter-length time series}, compute the Fourier transform (see \verb+fft+):
%
\begin{equation*}
H(f) = \cF[c(t)]
\end{equation*}
%
Check that $H(f)$ is complex and has the same number of points as $c(t)$.

%---------

\item In two different ways, show that $z^{*}z = r^2$, where $z = r e^{i\theta} = a + bi$.

%---------

\item  Compute the amplitude spectrum $A(f)$ (see \verb+conj+).
The power spectral density $P(f)$ and spectral amplitude $A(f)$ are real functions given by
%
\begin{eqnarray*}
P(f) &=& H^{*}(f)\,H(f)
\\
A(f) &=& |H(f)| = \sqrt{P(f)},
\end{eqnarray*}
%
where $H(f)$ is the Fourier transform of $c(t)$.

%---------

\item Plot the amplitude spectrum (use \verb+semilogy+) and describe it in qualitative terms. \\
Use the function \verb+fftvec()+ to obtain the frequency vector. \\
Plot the $y$-axis with the range \verb+ylim([1e-4 1e1])+.

%---------

\item  Use the function \verb+markp()+ to help identify periods associated with spectral peaks. (An example of how to use \verb+markp()+ will be done in lab.) Then answer the following questions:
%
\begin{enumerate}
\item What is the number of points $n$ in the discretized version of $c(t)$?
\item What is the number of points in the discretized version of $H(f)$?
\item What is the sampling interval $\Delta t$?
\item What is the Nyquist frequency $\fnyq$? \\
List an analytical expression and also the numerical value.
\item What is the frequency interval $\Delta f$?\footnote{This $\Delta f$ is {\em not} the sample rate, which is $1/\Delta t$ and is the same for both time series.} \\
List an analytical expression and also the numerical value.
\item What planet is retrograde\footnote{A retrograde planet will move in an opposite sense to the Sun's motion.}? (Fill out \refTab{tab:planets}.)
\item What planets are identifiable as prograde? (Fill out \refTab{tab:planets}.)
\item What planets are not resolved? (Fill out \refTab{tab:planets}.)
\end{enumerate}

\item Now try the {\bf longer-length time series} (\verb+longdata = 1+) and repeat the analysis.
%
\begin{enumerate}
\item What is the number of points $n$ in the discretized version of $c(t)$?
\item What is the sampling interval $\Delta t$?
\item What is the Nyquist frequency $\fnyq$?
\item What is the frequency interval $\Delta f$?
\item What planets are identifiable as prograde? (Fill out \refTab{tab:planets}.)
\item What planets are not resolved? (Fill out \refTab{tab:planets}.)
\end{enumerate}

\end{enumerate}

\end{spacing}

%-------------------------------------------------------------
%\bibliography{}
%-------------------------------------------------------------

\begin{table}[h]
\centering
\caption[]
{{
Summary table. P = prograde, R = retrograde, N = not resolved.
\label{tab:planets}
}}
\tgap
\begin{tabular}{|c|c|c|}
\hline
planet & short times series & long time series \\ \hline
Mercury & & \\ \hline
Venus & & \\ \hline
Earth & & \\ \hline
Mars & & \\ \hline
Jupiter & & \\ \hline
Saturn & & \\ \hline
Uranus & & \\ \hline
Neptune & & \\ \hline
Pluto & & \\ \hline
\end{tabular}
\end{table}

%-------------------------------------------------------------
\end{document}
%-------------------------------------------------------------
