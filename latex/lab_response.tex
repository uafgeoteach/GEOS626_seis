% dvips -t letter lab_response.dvi -o lab_response.ps ; ps2pdf lab_response.ps
\documentclass[11pt,titlepage,fleqn]{article}

\usepackage{amsmath}
\usepackage{amssymb}
\usepackage{latexsym}
\usepackage[round]{natbib}
%\usepackage{epsfig}
\usepackage{graphicx}
\usepackage{bm}

\usepackage{url}
\usepackage{color}
%\usepackage{hyperref}

%--------------------------------------------------------------
%       SPACING COMMANDS (Latex Companion, p. 52)
%--------------------------------------------------------------

\usepackage{setspace}    % double-space or single-space
\usepackage{xspace}

\renewcommand{\baselinestretch}{1.2}

\textwidth 460pt
\textheight 690pt
\oddsidemargin 0pt
\evensidemargin 0pt

% see Latex Companion, p. 85
\voffset     -50pt
\topmargin     0pt
\headsep      20pt
\headheight   15pt
\headheight    0pt
\footskip     30pt
\hoffset       0pt

\input{commands_letters}
\input{commands_uaf}
\input{commands_carl}

\newcommand{\blank}{xxxx}

\newcommand{\cyear}{2026}

% provide space for students to write their solutions
\newcommand{\vertgap}{\vspace{1cm}}

\graphicspath{
  {./figures/}
}

\newcommand{\cltag}{GEOS 626/426: Applied Seismology, Carl Tape}
\newcommand{\ptag}{{\bf \textcolor{magenta}{[GEOS 626]}}}
%\newcommand{\ptag}{}

\bibliographystyle{agufull08}

\newcommand{\howmuchtime}{Approximately how much time {\em outside of class and lab time} did you spend on this problem set? Feel free to suggest improvements here.}

%------------------------------------------------------

% modes
\newcommand{\tnl}[2]{\mbox{$_{#1}\ssT_{#2}$}}
\newcommand{\snl}[2]{\mbox{$_{#1}\ssS_{#2}$}}
\newcommand{\snlm}[3]{\mbox{$_{#1}\ssS_{#2}^{#3}$}}
\newcommand{\Tnl}{\mbox{${}_nT_l$}}   % eigenfun
\newcommand{\Wnl}{\mbox{${}_nW_l$}}   % eigenfun
\newcommand{\Ynl}{\mbox{${}_nY_l$}}   % spherical harmonics
% eigenfrequencies
\newcommand{\omnl}{\mbox{${}_{n\hspace{-0.2 mm}}\omega_l$}}
\newcommand{\fnl}{\mbox{${}_{n\hspace{-0.2 mm}}f_l$}}
\newcommand{\omnlm}[1]{\mbox{${}_{n\hspace{-0.2 mm}}\omega_l^{#1}$}}
\newcommand{\fnlm}[1]{\mbox{${}_{n\hspace{-0.2 mm}}f_l^{#1}$}}

\newcommand{\cutoff}[1]{{#1}_{x\bar{n}}}
\newcommand{\Wcol}{\textcolor{blue}{W}}
\newcommand{\Tcol}{\textcolor{red}{T}}

%\newcommand{\rfind}{{\tt fzero}}
\newcommand{\rfind}{{\tt brentq}}
%\newcommand{\sfind}{{\tt ode45}}
\newcommand{\sfind}{{\tt solve\_ivp}}
\newcommand{\maxstep}{{\tt max\_step}}

%------------------------------------------------------


\renewcommand{\baselinestretch}{1.05}

% change the figures to ``Figure L3'', etc
\renewcommand{\thefigure}{L\arabic{figure}}
\renewcommand{\thetable}{L\arabic{table}}
\renewcommand{\theequation}{L\arabic{equation}}
\renewcommand{\thesection}{L\arabic{section}}

%\newcommand{\fft}{h}
%\newcommand{\ffw}{\widetilde{h}}
\newcommand{\fft}{h}
\newcommand{\ffw}{H}

%\newcommand{\tfile}{{\tt CAN$\_$response$\_$template.m}}
\newcommand{\tfile}{{\tt lab\_response.ipynb}}

%\renewcommand{\thefigure}{\Alph{figure}}

%--------------------------------------------------------------
\begin{document}
%-------------------------------------------------------------

\begin{spacing}{1.2}
\centering
{\large \bf Lab Exercise: Instrument response [response]} \\
\cltag\ \\
%Assigned: February 13, 2014 --- Due: February 20, 2014 \\
Last compiled: \today
\end{spacing}

%------------------------

\section{Overview}

\begin{itemize}

\item You will use \tfile. The example station is CAN, which is in Canberra, Australia, and featured in \citet[][Figure~1]{Park2005}.

%---------

\item Background reading on instrument response and Fourier analysis: \\ \citet[][Ch.~6]{SteinWysession} and \verb+notes_fft.pdf+

Overview of where seismic data comes from: \citet{RinglerBastien2020}

%---------

%\item WARNING: This document is in a state of transition from Matlab to Python. The original lab used Matlab to generate the figures in this document. Your lab is in Python and will generate different figures using ObsPy \citep{obspy2010,obspy2011,obspy2015}. An excellent resource for learning ObsPy are the tutorials in seismolive (\url{www.seismo-live.org}) \citep{seismolive}.

%\item This lab and the associated problem set utilize the GISMO (``GI Seismology'') Waveform Suite for Matlab developed by Celso Reyes and Michael West \citep{ReyesWest2011} and now maintained by Glenn Thompson. It is available via github here:

%\verb+https://github.com/geoscience-community-codes/GISMO+

%GISMO is installed on the linux network and will not be available off the network computers (unless you install it yourself). For information, check out the wiki page:

%\verb+https://github.com/geoscience-community-codes/GISMO/wiki+

%In Matlab, type \verb+methods waveform+ to see the main functions in the toolbox.

%---------

%\item There are different tools in seismology for deconvolving instrument responses and for processing seismic waveforms. We are using Matlab and GISMO, but alternative tools include ObsPy \citep{obspy2010} and SAC \citep{SAC}. {\bf Do not get bogged down in the details of the functions we are using.} Our goal is to understand how to {\em use} these tools.

\item There is no seismic data in this lab. Metadata only (in particular, the response of a sensor).

%---------

\item Your goal is to understand how an instrument response is defined (\eg as a set of poles and zeros of a complex function), how it is plotted (\eg as phase and amplitude, each varying with frequency), and how it depends on displacement, velocity, and acceleration.

%---------

\item Aside: Two seminal advances in seismometry were the creation of the forced-feedback seismometer \citep{WielandtStreckheisen1982} and a modified feedback circuit to enable very broadband capabilities \citep{WielandtSteim1986}. The seismometer featured in our lab is an STS-1, which is named after Streckheisen, who started a company to build seismometers.

\end{itemize}

%------------------------

%\pagebreak
\section{Background}

The output from a seismometer is in ``counts'' as a function of time. (See Note \footnote{The IRIS webpage for frequently asked questions says: ``Counts'' is the raw number read off the physical instrument, ie. the voltage read from a sensor.}.) The scaling from counts to ground motion is not simple and is described by the ``instrument response''.

The relationship between ground motion, instrument response, and the output from a seismometer can be written in the time domain or frequency domain as
%
\begin{eqnarray}
x_a(t) * i_a(t) &=& c(t)
\\
X_a(\omega) I_a(\omega) &=& C(\omega)
\end{eqnarray}
%
Here we have assumed that the instrument response is described with respect to acceleration. But we could alternatively consider the instrument response with respect to velocity or displacement; {\bf the key point is that what comes out of the seismometer, $c(t)$, is fixed.}

Instrument response is a complex function that can be written as
%
\begin{equation}
I(\omega) = A(\omega) e^{i \phi(\omega)}
\label{Iw}
\end{equation}
%
For most theoretical derivations, $\omega = 2\pi f$ tends to simplify the notation. However, for interpretation purposes, $f$ (or period, $T = 1/f$) is a more intuitive quantity. Most of our plots will use $f$, not $\omega$.

%------------------------

\pagebreak
\section{Computing and plotting instrument response, $I(\omega)$}

Run \tfile, which produces \refFigab{fig1}{fig3}\footnote{The versions of the figures in \refFigab{fig1}{fig3} were generated in Matlab and differ somewhat from those that appear in the Python notebook}. {\bf Examine the figures and read the code in order to understand what is being done and shown.}
%
\begin{itemize}
\item Start with \refFig{fig1}, which shows the instrument response for G.CAN.BHZ., which means network G (Geoscope), station CAN (Canberra, Australia), channel BHZ (B = 20 samples per second, H = broadband sensor, Z = vertical component), and empty location code.
%
%\refFig{fig1} is generated with two commands [update to python needed]
%
%\begin{verbatim}
%res0 = response_get_from_db(station,channel,startTime,f,dbname);
%response_plot(res0);
%\end{verbatim}
``%
%This uses utilities within GISMO to read the instrument response within the Antelope database.
Also needed to get the instrument response are a time, a vector of frequencies, and the name of the database. The time (\verb+starttime+) is needed because the instrument response may change as a function of time, based on changes made at the station, either from in-person visits or from remote interaction. The instrument response file could change because, for example, maybe the sensor is replaced, or it is reoriented, or the response is updated from what the manufacturer provided.

The plot shows $\phi(\omega)$ and $A(\omega)$, as in \refEq{Iw}.

The sample rate for this channel is 20~samples per second ($\Delta t = 0.05$~s), so the {\bf Nyquist frequency}
%
\begin{equation}
\fnyq = 1/(2\Delta t)
\end{equation}
%
is $\fnyq = 10$~Hz.

%---------------------

\item \refFig{fig2} shows the instrument response to displacement, velocity, and acceleration, denoted by $I_d(\omega)$, $I_v(\omega)$, and $I_a(\omega)$. We will see in the homework that 
%
\begin{eqnarray*}
X_v(\omega) &=& (i\omega) X_d(\omega)
\\
X_a(\omega) &=& (i\omega) X_v(\omega)
\end{eqnarray*}
%
where $X_d$, $X_v$, and $X_a$ are the Fourier transforms of displacement $x_d(t)$, velocity $x_v(t)$, and acceleration $x_a(t)$.

We can describe the input ground motion as displacement, velocity, or acceleration. Showing all three together and omitting explicit $\omega$ dependence, we have
%
\begin{eqnarray}
C &=& X_a I_a = X_v I_v =  X_d I_d 
\\
&=& (i\omega) X_v I_a = (i\omega) X_d I_v =  X_d I_d 
\\
&=& (i\omega)^2 X_d I_a = (i\omega) X_d I_v =  X_d I_d 
\\
I_v &=& I_d / (i\omega)
\\
I_a &=& I_v / (i\omega)
\end{eqnarray}
%
It turns out that the effect of differentiation in the time domain leads to an {\em increase by a factor of one} in the slope of the amplitude spectrum of ground motion ($H(\omega)$) in log-log space, for example, by changing from $X_d(\omega)$ to $X_v(\omega)$. But when we are looking at the {\em instrument response}, the slope will {\em decrease by a factor of one} when changing from, say, $I_d(\omega)$ to $I_v(\omega)$.

We see this in \refFig{fig2}. Consider the flat segment in $I_v(\omega)$. Change to $I_a(\omega)$ and the slope decreases; change to $I_d(\omega)$ and the slope increases.

%-------------

\item \refFig{fig3} shows the full instrument response, including the digitizer+FIR (finite impulse response).
%This is the classical ``simple'' plot of a pole-zero response. This pole-zero file is shown in \refApp{sec:pz}.

For $f > \fnyq = 0.5$~Hz we see that the phase response is quite complicated (this is the finite-impulse response filter), and the amplitude response declines to $10^{-5}$. As a general rule, you need to be very careful if you are looking at signals with frequencies above $\fnyq$.


\end{itemize}

%------------------------

%\pagebreak
\subsection*{Questions}

\begin{enumerate}
\item Mark the Nyquist frequency in the plots in \refFig{fig1}.

\item What are the variables \verb+Id+, \verb+Iv+, and \verb+Ia+?

What do the functions \verb+angle+ and \verb+abs+ do to these?

\item What is meant by a ``broadband'' seismometer?

\item Change the channel to LHZ and re-run the notebook. How do the plots change?

\item Referring to the pole-zero file (\refApp{sec:pz}), what are the input and output units for this instrument response?

\item Referring to the pole-zero file (\refApp{sec:pz}), what is the time interval over which this instrument response is valid?


\end{enumerate}

%------------------------

\iffalse
\section{Deconvolve instrument response for a seismogram}

FUTURE PART OF THE LAB (see also \verb+lab_sumatra.pdf+):
%
\begin{verbatim}
otime = 2004-12-26 00:58:50 = 732307.040856 [matlab days]
station CAN.G
duration = 10 days
start time: t1 = otime - duration - 1
end time: t2 = t1 + duration
channels: LHZ, LHE, LHN

Deconvolve instrument response on all three components.
Plot the amplitude spectrum over 0.2--1.0 mHz to show the gravest mode peaks.
Show what the main arrival looks like at bandpass 50-500s with and without decon.
Rotate to R and T to isolate SH waves.
[Having the previous 10 days allows for the noise analysis in hw_sumatraA.pdf]
\end{verbatim}
\fi

%-------------------------------------------------------------
%\pagebreak
\bibliography{uaf_abbrev,uaf_main,uaf_carletal}
%-------------------------------------------------------------

\pagebreak

\appendix

\section{Example pole-zero file}
\label{sec:pz}

This is an example pole-zero file retrieved from IRIS. While some programs might use pole-zero files in data processing, we do not, since all of this information is saved in memory when we fetch it from IRIS.

\small
\begin{spacing}{1.0}
\begin{verbatim}
* **********************************
* NETWORK   (KNETWK): G
* STATION    (KSTNM): CAN
* LOCATION   (KHOLE): 
* CHANNEL   (KCMPNM): LHZ
* CREATED           : 2012-03-26T22:39:01
* START             : 1989-06-02T00:00:00
* END               : 2006-12-10T02:60:60
* DESCRIPTION       : Canberra, Australia
* LATITUDE          : -35.321000
* LONGITUDE         : 148.999000 
* ELEVATION         : 650.0  
* DEPTH             : 0.0  
* DIP               : 0.0  
* AZIMUTH           : 0.0  
* SAMPLE RATE       : 1.0
* INPUT UNIT        : M
* OUTPUT UNIT       : COUNTS
* INSTTYPE          : STRECKEISEN STS1
* INSTGAIN          : 2.252000e+03 (M/S)
* COMMENT           : N/A
* SENSITIVITY       : 1.844840e+09 (M/S)
* A0                : 3.727650e+12
* **********************************
ZEROS	3
	+0.000000e+00	+0.000000e+00
	+0.000000e+00	+0.000000e+00
	+0.000000e+00	+0.000000e+00
POLES	10
	-1.233947e-02	+1.234318e-02
	-1.233947e-02	-1.234318e-02
	-3.917563e+01	+4.912335e+01
	-3.917563e+01	-4.912335e+01
	-3.034990e+01	+7.868112e+00
	-3.034990e+01	-7.868112e+00
	-2.220727e+01	+2.208852e+01
	-2.220727e+01	-2.208852e+01
	-8.135964e+00	+3.016618e+01
	-8.135964e+00	-3.016618e+01
CONSTANT	+6.876917e+21
\end{verbatim}
\end{spacing}
\normalsize

%-------------------------------------------------------------

%\clearpage

%\begin{table}[h]
\caption[Wave parameters and some related equations]
{{
Wave parameters and some related equations.
Note that $\lim_{l\rightarrow \infty } \sqrt{l(l+1)} = l + \hlf$ (see {\tt notes\_wave\_params.pdf}), so in the high-frequency limit, $l + \hlf$ is appropriate.
We have listed ``rad'' for radians, which have no units; however, it is helpful to think of angles for these quantities.
%The equations in the last column incorporate $l$ and $a$ and are approximate for $l \gg 1$ (see \citet{Woodhouse1996}, Eq.~2.110; \citet{DH1986}).
\label{tab:waveparm}
}}
\hspace{-1cm}
\begin{tabular}{||c|c|c|c|c||}
\hline
%
%& & & \\
% ADJUST THE WIDTH OF THE COLUMNS HERE
earth radius & \hspace{10pt} $a$ \hspace{10pt}
& $6.371 \times 10^6$ m & & \\
\cline{1-3}
earth circumference & $2\pi a$ & $4.003 \times 10^7$ m & (with $l$) & (with $l$ and $a$)  \\
%& & & \\
\cline{1-3}
%
%& & & \\
degree	& $l$ & $0,1,2,\ldots$ & & \\
%& & & \\
\hline
%
& & & & \\
period		& $T$
& $T = \frac{1}{f} = \frac{2\pi}{\omega} = \frac{\lambda}{c} = \frac{2\pi}{c k}$ 
& $T = \frac{2\pi}{c\sqrt{l(l+1)}}$ 
& $T = \frac{2\pi a}{c'\sqrt{l(l+1)}}$
\\
& & & [s] & [s] \\ \hline
%
& & & & \\
frequency	& $f$
& $f = \frac{\omega}{2\pi} = \frac{1}{T} = \frac{c}{\lambda} = \frac{ck}{2\pi}$ 
& $f = \frac{c\sqrt{l(l+1)}}{2\pi}$
& $f = \frac{c'\sqrt{l(l+1)}}{2\pi a}$
\\
& & & [1/s] & [1/s] \\ \hline
%
& & & & \\
angular frequency	& $\omega$		
& $\omega = 2\pi f = \frac{2\pi}{T} = ck = \frac{2\pi c}{\lambda}$
& $\omega = c\sqrt{l(l+1)}$
& $\omega = \frac{\;c'\sqrt{l(l+1)}\;}{a}$
\\
& & & [rad/s] & [rad/s] \\ \hline\hline
%
& & & & \\
wavelength	& $\lambda$		
& $\lambda = \frac{c}{f} = c\,T = \frac{2\pi}{k} = \frac{2\pi c}{\omega}$
& $\lambda = \frac{2\pi}{\sqrt{l(l+1)}}$
& $\lambda' = \frac{2\pi a}{\sqrt{l(l+1)}} = \lambda\,a$
\\
& & & [rad] & [m] \\ \hline		
%
& & & & \\
wavenumber		& $k$	
& $k = \frac{\omega}{c} = \frac{2\pi}{\lambda} = \frac{2\pi f}{c} = \frac{2\pi}{Tc}$
& $k = \sqrt{l(l+1)}$
& $k' = \frac{\sqrt{l(l+1)}}{a} = k/a$
\\
& & & [] & [1/m] \\ \hline
%
& & & & \\
phase speed		& $c$			
& $c = \frac{\omega}{k} = \frac{\lambda}{T} = f\lambda = \frac{2\pi f}{k} = \frac{2\pi}{kT}$
& $c = \frac{2\pi}{\;T\sqrt{l(l+1)}\;}$
& $c' = \frac{2\pi a}{\;T\sqrt{l(l+1)}\;} = c\,a$
\\
& & & [rad/s] & [m/s] \\ \hline
& & & & \\
group speed             & $U$
& $U = \frac{d\omega}{d k} = \frac{1}{2\pi}\frac{d f}{d k} = k + \frac{dc}{dk} = c - \frac{dc}{d\lambda}$
& $U$ 
& $U' = U\,a$
\\
& & & [rad/s] & [m/s] \\
\hline
\end{tabular}
\end{table}


%\setcounter{figure}{2}

\begin{figure}
\centering
\includegraphics[width=16cm]{CAN_response_fig3.eps}
\caption[]
{{
Output from \tfile.
Note that the amplitudes are normalized by the maximum value.
}}
\label{fig1}
\end{figure}

\begin{figure}
\centering
\includegraphics[width=16cm]{CAN_response_fig5.eps}
\caption[]
{{
Output from \tfile.
The dashed red lines are plotted on top of the blue lines as a check.
}}
\label{fig2}
\end{figure}

\begin{figure}
\centering
\includegraphics[width=16cm]{CAN_response_fig1.eps}
\caption[]
{{
Output from \tfile, for the case of LHZ, showing the FIR filter that is present above the Nyquist frequency.
Here the amplitudes are normalized by the maximum value.
}}
\label{fig3}
\end{figure}

%-------------------------------------------------------------
\end{document}
%-------------------------------------------------------------
