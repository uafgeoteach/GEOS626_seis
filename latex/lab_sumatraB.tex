% dvips -t letter lab_sumatra.dvi -o lab_sumatra.ps ; ps2pdf lab_sumatra.ps
\documentclass[11pt,titlepage,fleqn]{article}

\usepackage{amsmath}
\usepackage{amssymb}
\usepackage{latexsym}
\usepackage[round]{natbib}
%\usepackage{epsfig}
\usepackage{graphicx}
\usepackage{bm}

\usepackage{url}
\usepackage{color}
%\usepackage{hyperref}

%--------------------------------------------------------------
%       SPACING COMMANDS (Latex Companion, p. 52)
%--------------------------------------------------------------

\usepackage{setspace}    % double-space or single-space
\usepackage{xspace}

\renewcommand{\baselinestretch}{1.2}

\textwidth 460pt
\textheight 690pt
\oddsidemargin 0pt
\evensidemargin 0pt

% see Latex Companion, p. 85
\voffset     -50pt
\topmargin     0pt
\headsep      20pt
\headheight   15pt
\headheight    0pt
\footskip     30pt
\hoffset       0pt

\input{commands_letters}
\input{commands_uaf}
\input{commands_carl}

\newcommand{\blank}{xxxx}

\newcommand{\cyear}{2026}

% provide space for students to write their solutions
\newcommand{\vertgap}{\vspace{1cm}}

\graphicspath{
  {./figures/}
}

\newcommand{\cltag}{GEOS 626/426: Applied Seismology, Carl Tape}
\newcommand{\ptag}{{\bf \textcolor{magenta}{[GEOS 626]}}}
%\newcommand{\ptag}{}

\bibliographystyle{agufull08}

\newcommand{\howmuchtime}{Approximately how much time {\em outside of class and lab time} did you spend on this problem set? Feel free to suggest improvements here.}

%------------------------------------------------------

% modes
\newcommand{\tnl}[2]{\mbox{$_{#1}\ssT_{#2}$}}
\newcommand{\snl}[2]{\mbox{$_{#1}\ssS_{#2}$}}
\newcommand{\snlm}[3]{\mbox{$_{#1}\ssS_{#2}^{#3}$}}
\newcommand{\Tnl}{\mbox{${}_nT_l$}}   % eigenfun
\newcommand{\Wnl}{\mbox{${}_nW_l$}}   % eigenfun
\newcommand{\Ynl}{\mbox{${}_nY_l$}}   % spherical harmonics
% eigenfrequencies
\newcommand{\omnl}{\mbox{${}_{n\hspace{-0.2 mm}}\omega_l$}}
\newcommand{\fnl}{\mbox{${}_{n\hspace{-0.2 mm}}f_l$}}
\newcommand{\omnlm}[1]{\mbox{${}_{n\hspace{-0.2 mm}}\omega_l^{#1}$}}
\newcommand{\fnlm}[1]{\mbox{${}_{n\hspace{-0.2 mm}}f_l^{#1}$}}

\newcommand{\cutoff}[1]{{#1}_{x\bar{n}}}
\newcommand{\Wcol}{\textcolor{blue}{W}}
\newcommand{\Tcol}{\textcolor{red}{T}}

%\newcommand{\rfind}{{\tt fzero}}
\newcommand{\rfind}{{\tt brentq}}
%\newcommand{\sfind}{{\tt ode45}}
\newcommand{\sfind}{{\tt solve\_ivp}}
\newcommand{\maxstep}{{\tt max\_step}}

%------------------------------------------------------


% change the figures to ``Figure L3'', etc
\renewcommand{\thefigure}{L\arabic{figure}}
\renewcommand{\thetable}{L\arabic{table}}
\renewcommand{\theequation}{L\arabic{equation}}
\renewcommand{\thesection}{L\arabic{section}}

%\newcommand{\tfile}{{\tt sumatra\_modes.m}}
\newcommand{\tfile}{{\tt lab\_sumatraB.ipynb}}
\newcommand{\tfilemodes}{{\tt hw\_sumatra\_modes.ipynb}}

\renewcommand{\baselinestretch}{1.0}

%--------------------------------------------------------------
\begin{document}
%-------------------------------------------------------------

\begin{spacing}{1.2} 
\centering
{\large \bf Lab Exercise: Waveforms from the 2004 Sumatra-Andaman earthquake [sumatraB]} \\
\cltag\ \\
Last compiled: \today
\end{spacing}
%------------------------

\subsection*{Overview and Instructions}

\begin{itemize}
\item The key file is \tfile.

\item This lab is a starter for the homework on the Sumatra earthquake (\verb+hw_sumatraB+). Here we download and screen data in preparation for the homework.

The goal is to understand the attributes of a seismic waveform (start time, end time, sample rate, station location, etc) and to gain some appreciation for variability in the quality of waveforms, especially when recording a \magw{9} earthquake, which puts some seismometers to a difficult test!

%---------

\item {\bf Tips on picking stations for analysis}:
\begin{itemize}
\item Pick time series that have the highest signal-to-noise ratios.
\item In several cases there are multiple seismograms for the same station (and component), \eg at the South Pole\footnote{You definitely want to keep a QSPA spectrum for your analysis.}:
%
\begin{spacing}{1.0}
\begin{verbatim}
IU.QSPA.00.LHZ
IU.QSPA.10.LHZ
IU.QSPA.20.LHZ
IU.QSPA.30.LHZ
\end{verbatim}
\end{spacing}
%
%(You can see these lines when running \verb+sumatra_hf.ipynb+.)
You do not need more than one seismogram for each site. In general the multiple listings represent different seismometers being used at different vertical locations, like at the surface and down a deep borehole. In fact, different networks might have a station at the same location.

\item For such a big event as Sumatra, there are many seismograms that are ``clipped'' or distorted due to an erroneous response of the seismometer.

\item Having a uniform distribution (by distance, azimuth, latitude, etc) is often needed for analyses. {\bf So when you are picking a subset, be sure you have enough stations to cover the particular variation of interest.} Look over the homework problems to see what variations we will be covering. Use the global map of stations for help in picking.
\end{itemize}

\item We select stations from three seismic networks: \\

\begin{tabular}{cll}
\hline
G   & Geoscope      & Institut de Physique du Globe de Paris \\
II  & GSN-IRIS/IDA  & IRIS and Scripps Institute of Oceanography \\
IU  & GSN-IRIS/USGS & IRIS and USGS Albuquerque Seismological Laboratory \\
\hline
\end{tabular} \\

\noindent
II and IU form the Global Seismographic Network (GSN).
The Geoscope network is managed by France and has been running high-quality stations since 1982.

\item The time periods for the waveforms are specified with respect to the origin time of the Sumatra earthquake: 2004--12--26 00:58:53

%\noindent
%This is the PDE (Preliminary Determination of Epicenter from National Earthquake Information Center) origin time listed in the CMTSOLUTION file (\url{www.globalcmt.org}).
%The decimal version gives the time in days with respect to some Matlab reference time. In the lists below, all times and durations are in days.

\end{itemize}

%------------------------

%\pagebreak
\subsection*{Exercises}

\begin{enumerate}
\item Examine \tfile. For the LHZ raw data, list the parameters specified in the data download request:

\begin{itemize}
\item networks: 
\item stations: 
\item locations: 
\item channels: 
\item start time: 
\item end time: 
\item duration: 
\end{itemize}

Note that \verb+remove_response+ is set to \verb+False+, so these waveforms are in units of counts.

%\item Notice that \verb+sumatra_loop.py+ is called from \tfile. Try to determine what \verb+sumtra_loop.py+ is doing. Notice that it will write the file \verb+sumatra_modes.txt+ to the directory \verb+./datawf/sumatra_fft/+.

%==================================================

\item Run the relevant cells in \tfile\ to fetch the database of raw LHZ waveforms.
%If successful, you should see a large set of seismograms displayed.

Your seismograms will be saved in the folder \verb+datawf/sumatra_LHZ_raw/+.

\item Explain the four parts that make up each waveform ID. (See also \verb+lab_record_section.pdf+.) Note that the order of the four parts is critical.

\item In the relevant code cell, set the flag \verb+plot_seismogram+ to \verb+True+ and examine the seismograms. Describe three ``errors'' in the seismograms that should probably be rejected from the analysis.

%Open \verb+sumatra_modes.txt+ and make sure you understand what is shown.

\item In the relevant code cell, set the flag \verb+plot_spectra+ to \verb+True+ and browse through all seismograms and their corresponding amplitude spectra together. Pick a subset of {\bf at least 20 stations} for the analysis of normal modes. (The more the better!)
%Pick only one location for a given station.
%You can identify the subset by the list of indices in \verb+ipick+.

%From a terminal window (not Matlab), use \verb+xpdf+ or \verb+evince+ to browse the 139-page pdf file
%
%\verb+/home/admin/databases/SUMATRA/data/wfobject/all_sumatra_modes.pdf+
%
%to pick a subset of {\bf at least 20 stations} for the analysis of normal modes. Write down the page numbers of the stations you want to keep.
The stations should be selected based on high signal-to-noise within the frequency range 0.2 to 1.0~mHz, and, notably, the \snl{0}{2} peak(s). Be sure to include Canberra and the station from Figure~2 of \citet{Park2005}.
{\bf See ``Tips on picking stations'' in the instructions above.}

\item Run relevant cells to make a source-station plot for the selected stations and a table for the station distances and azimuths.

\item Run the manual response removal on your selected subset of seismograms and analyze their spectra.
These seismograms will have the response removed to acceleration.
%Your seismograms will be saved locally in the folder \verb+datawf/sumatra_LHZ_acc/+.
% ZZZ Acceleration spectra are preferred output for analyzing normal modes.

\end{enumerate}

%------------------------

\iffalse

%\pagebreak
\section{Building the database of waveforms [FOR READING ONLY; THIS IS NOW OBSOLETE, SINCE EACH STUDENT IS CREATING THEIR OWN DATABASE OF WAVEFORMS]}

I have set up a database of waveforms for the Sumatra earthquake so that you can easily and rapidly access the waveforms. Most seismic waveforms can be obtained from data centers such as the IRIS DMC (\url{http://ds.iris.edu/ds/nodes/dmc/}). In a future version of this lab, I hope that students will download the waveforms themselves.

\subsection{Fetching waveforms}

\medskip\noindent
The time periods for the waveforms are specified with respect to the origin time of the Sumatra earthquake:

\vspace{0.5cm}
% otimePDE = datenum(2004,12,26,0,58,50); sprintf('%.6f',otimePDE)
otimePDE = 2004--12--26 00:58:50 = 732307.040856
\vspace{0.5cm}

\noindent
This is the PDE (Preliminary Determination of Epicenter from National Earthquake Information Center) origin time listed in the CMTSOLUTION file (\url{www.globalcmt.org}). The decimal version gives the time in days with respect to some Matlab reference time. In the lists below, all times and durations are in days.

\medskip\noindent
We select stations from three seismic networks: \\

\begin{tabular}{cll}
\hline
G   & Geoscope      & Institut de Physique du Globe de Paris \\
II  & IRIS/IDA      & Scripps Institute of Oceanography \\
IU  & GSN-IRIS/USGS & Albuquerque Seismological Laboratory \\
\hline
\end{tabular} \\

\noindent
II and IU form the Global Seismographic Network (GSN).
The Geoscope network is managed by France and has been running high-quality stations since 1982.

\bigskip\noindent
There are three sets of waveforms within our Sumatra database:
%
\begin{enumerate}
\item {\bf earthquake waveforms: high sample-rate}
\begin{itemize}
\item all stations in networks G, II, IU
\item duration = 5/24 days (5 hours)
\item start time: t1 = otimePDE - 1/24
\item end time: t2 = t1 + duration
\item channels: BH?
\end{itemize}

\item {\bf earthquake and modes: low sample-rate}
\begin{itemize}
\item all stations in networks G, II, IU
\item duration = 10 days
\item start time: t1 = otimePDE - duration/20
\item end time: t2 = t1 + duration
\item channels: LH?
\end{itemize}

\item {\bf pre-earthquake noise: low sample-rate}
\begin{itemize}
\item all stations in network G
\item duration = 10 days
\item start time: t1 = otimePDE - 1 - duration
\item end time: t2 = t1 + duration
\item channels: LH?
\end{itemize}

\end{enumerate}

%\pagebreak\noindent
If the size of the data request is not an issue, then it would be simpler to make two requests:
%
\begin{enumerate}
\item {\bf earthquake waveforms: high sample-rate}

\begin{itemize}
\item all stations in networks G, II, IU
\item duration = 5/24 days (5 hours)
\item start time: t1 = otimePDE - 1/24
\item end time: t2 = t1 + duration
\item channels: BH?
\end{itemize}

\item {\bf earthquake and modes: low sample-rate}
\begin{itemize}
\item all stations in networks G, II, IU
\item duration = 10 days
\item start time: t1 = otimePDE - 1 - duration
\item end time: t2 = otimePDE + duration
\item channels: LH?
\end{itemize}

\end{enumerate}

With these parameters, we can make a command-line request to IRIS DMC to obtain the waveforms. Tools such as ObsPy can be used as well \citep{obspy2010,obspy2011}.

%-------------------------------------------------------------

\subsection{Building a local database of waveforms}

Once the waveforms are obtained, it is helpful to store them as a database. We used Antelope to construct the database of stations, waveforms, and instrument response files. For your homework exercises, you will use Matlab scripts to extract seismic waveforms from the Antelope database. It is important to understand the two-part process of obtaining waveforms (and response files) and creating a database.

\fi

%-------------------------------------------------------------
%\pagebreak
\bibliography{uaf_abbrev,uaf_carletal,uaf_main,uaf_slab}
%-------------------------------------------------------------

%-------------------------------------------------------------
\end{document}
%-------------------------------------------------------------
