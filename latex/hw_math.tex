% dvips -t letter hw_math.dvi -o hw_math.ps ; ps2pdf hw_math.ps
\documentclass[11pt,titlepage,fleqn]{article}

\usepackage{amsmath}
\usepackage{amssymb}
\usepackage{latexsym}
\usepackage[round]{natbib}
%\usepackage{epsfig}
\usepackage{graphicx}
\usepackage{bm}

\usepackage{url}
\usepackage{color}
%\usepackage{hyperref}

%--------------------------------------------------------------
%       SPACING COMMANDS (Latex Companion, p. 52)
%--------------------------------------------------------------

\usepackage{setspace}    % double-space or single-space
\usepackage{xspace}

\renewcommand{\baselinestretch}{1.2}

\textwidth 460pt
\textheight 690pt
\oddsidemargin 0pt
\evensidemargin 0pt

% see Latex Companion, p. 85
\voffset     -50pt
\topmargin     0pt
\headsep      20pt
\headheight   15pt
\headheight    0pt
\footskip     30pt
\hoffset       0pt

\input{commands_letters}
\input{commands_uaf}
\input{commands_carl}

\newcommand{\blank}{xxxx}

\newcommand{\cyear}{2026}

% provide space for students to write their solutions
\newcommand{\vertgap}{\vspace{1cm}}

\graphicspath{
  {./figures/}
}

\newcommand{\cltag}{GEOS 626/426: Applied Seismology, Carl Tape}
\newcommand{\ptag}{{\bf \textcolor{magenta}{[GEOS 626]}}}
%\newcommand{\ptag}{}

\bibliographystyle{agufull08}

\newcommand{\howmuchtime}{Approximately how much time {\em outside of class and lab time} did you spend on this problem set? Feel free to suggest improvements here.}

%------------------------------------------------------

% modes
\newcommand{\tnl}[2]{\mbox{$_{#1}\ssT_{#2}$}}
\newcommand{\snl}[2]{\mbox{$_{#1}\ssS_{#2}$}}
\newcommand{\snlm}[3]{\mbox{$_{#1}\ssS_{#2}^{#3}$}}
\newcommand{\Tnl}{\mbox{${}_nT_l$}}   % eigenfun
\newcommand{\Wnl}{\mbox{${}_nW_l$}}   % eigenfun
\newcommand{\Ynl}{\mbox{${}_nY_l$}}   % spherical harmonics
% eigenfrequencies
\newcommand{\omnl}{\mbox{${}_{n\hspace{-0.2 mm}}\omega_l$}}
\newcommand{\fnl}{\mbox{${}_{n\hspace{-0.2 mm}}f_l$}}
\newcommand{\omnlm}[1]{\mbox{${}_{n\hspace{-0.2 mm}}\omega_l^{#1}$}}
\newcommand{\fnlm}[1]{\mbox{${}_{n\hspace{-0.2 mm}}f_l^{#1}$}}

\newcommand{\cutoff}[1]{{#1}_{x\bar{n}}}
\newcommand{\Wcol}{\textcolor{blue}{W}}
\newcommand{\Tcol}{\textcolor{red}{T}}

%\newcommand{\rfind}{{\tt fzero}}
\newcommand{\rfind}{{\tt brentq}}
%\newcommand{\sfind}{{\tt ode45}}
\newcommand{\sfind}{{\tt solve\_ivp}}
\newcommand{\maxstep}{{\tt max\_step}}

%------------------------------------------------------


\renewcommand{\baselinestretch}{1.1}

%--------------------------------------------------------------
\begin{document}
%-------------------------------------------------------------

\begin{spacing}{1.2}
\centering
{\large \bf Problem Set 2: Linear algebra and vectors [math]} \\
\cltag\ \\
Assigned: January 21, \cyear\ --- Due: January 28, \cyear\ \\
Last compiled: \today \\
\end{spacing}

%------------------------

\subsection*{Overview}

The purpose of this problem set is to review a few mathematical concepts that are important to seismology. These concepts will be useful in following the class reading \citep{SteinWysession,ShearerE2} and for future homework assignments.

{\bf Except for Problem 1}, you are welcome to use a computer (Python, Matlab, etc) to support your answers. If you find yourself doing horrendous algebra, then probably you have made a prior mistake.

%------------------------

\subsection*{Problem 1 (5.0). Matrix decompositions}

This problem must be done by hand, unless specified. ``By hand'' means showing your steps and not using decimal notation; for example, write the expression $(4 +\sqrt{3})/5$ rather than 1.1464.
Be as clear as possible in explaining what equations you are using.
%You are welcome to check your answers using Matlab.

\medskip\noindent
The matrix for this problem is
%
\begin{equation*}
\bA =  \left[ \begin{array}{rrr}
    -1  &   0  &  -4 \\
     0  &   2  &   1 \\
     1  &   1  &   4 \\
\end{array} \right]
\end{equation*}

\input{hw_math_input}

%------------------------

\subsection*{Problem 2 (3.0). Matrix operations}

\begin{enumerate}

\item (0.3) Determine whether the following three vectors are linearly dependent:
%
\begin{eqnarray*}
\left[ \begin{array}{r} 1 \\ -2 \\ 4  \end{array} \right],
\left[ \begin{array}{r} 0 \\ 1 \\ 1 \end{array} \right],
\left[ \begin{array}{r} 1 \\ 4 \\ -2 \end{array} \right]
\end{eqnarray*}

%----------------

\item (0.3) Find the matrix $\bA$ ($T(\bx) = \bA\bx$) of the following linear transformations:
%
\begin{enumerate}
\item $\mathbb{R}^2 \rightarrow \mathbb{R}^2$ reflection about $y = 0$.
\item $\mathbb{R}^2 \rightarrow \mathbb{R}^2$ reflection about $x = 0$.
\item $\mathbb{R}^3 \rightarrow \mathbb{R}^3$ reflection through $xy$-plane.
\end{enumerate}
%
Hint: What does $\bA$ do to the input vector $(x,y)$?

%----------------

\item (0.3)
%You can use Matlab for any tedious steps.

\begin{enumerate}
\item (0.1) Find the matrix, $\bA$, ($T(\bx) = \bA\bx$) of the linear transformation described by
%
\begin{equation*}
T \left( \left[ \begin{array}{r} 1 \\ 0 \\ 0  \end{array} \right] \right) = \left[ \begin{array}{r} 1 \\ 2 \end{array} \right],
\hspace{0.5cm}
T \left( \left[ \begin{array}{r} 0 \\ 1 \\ 0  \end{array} \right] \right) = \left[ \begin{array}{r} 2 \\ 4 \end{array} \right],
\hspace{0.5cm}
T \left( \left[ \begin{array}{r} 0 \\ 0 \\ 1  \end{array} \right] \right) = \left[ \begin{array}{r} 3 \\ 6 \end{array} \right]
\end{equation*}

\item (0.2) Find the matrix, $\bA$, ($T(\bx) = \bA\bx$) of the linear transformation described by
%
\begin{equation*}
T \left( \left[ \begin{array}{r} 1 \\ 0 \\ 0  \end{array} \right] \right) = \left[ \begin{array}{r} 1 \\ 1 \\ 0 \end{array} \right],
\hspace{0.5cm}
T \left( \left[ \begin{array}{r} 2 \\ 1 \\ 0  \end{array} \right] \right) = \left[ \begin{array}{r} 1 \\ 0 \\ -1 \end{array} \right],
\hspace{0.5cm}
T \left( \left[ \begin{array}{r} 3 \\ 2 \\ 1  \end{array} \right] \right) = \left[ \begin{array}{r} 0 \\ 1 \\ 1 \end{array} \right]
\end{equation*}

\end{enumerate}

%----------------

\pagebreak
\item (0.8) 
\begin{enumerate}
\item (0.2) Compute the angle between $\bu = (1,1,1)^T$ and $\bv = (1,2,3)^T$.

\item Compute the area of the parallelogram defined by these vectors using three different methods:
\begin{enumerate}
\item (0.2) Use the height vector, $\bh = \bv - \proj_{\ssL} \bv$, where $\ssL$ is the line in the direction of $\bu$ (\refFig{fig:proj}).
%Sketch the various vectors and the parallelogram.
\item (0.2) Use the cross product formula, $\| \bu \times \bv \|$ (note: {\em not} the version $\|\bu\|\|\bv\|\sin\theta$, which is used in part a).
\item (0.2) Compute $\sqrt{\det(\bA^T \bA)}$, where $\bA = [\;\bu \; \bv\;]$.
\end{enumerate}

\end{enumerate}

%----------------

\item (0.3) \citep[][p.~473, P10]{SteinWysession}.

Prove that the magnitude of a vector is preserved by an orthogonal transformation.

Hint: Let $A\bu = \bv$ and show that $\|\bu\| = \|\bv\|$.

%----------------

\item (0.6) Show that 
%
\begin{eqnarray}
\hlf \left(\bG\bem - \bd \right)^T \left(\bG\bem - \bd\right)
\end{eqnarray}
%
can be written as 
%
\begin{eqnarray}
a + \bb^T\bem + \hlf \bem^T\bC\bem,
\end{eqnarray}
%
where
%
\begin{equation}
a = \hlf\bd^T\bd
\,,\hspace{20pt}
\bb = -\bG^T\bd
\,,\hspace{20pt}
\bC = \bG^T\bG
\,.
\end{equation}
%
Hint: For a scalar $a$, note that $a = a^T$.

%----------------

\item (0.4) Find all $\bx$ such that $\bA\bx = \bb$, where $\bb = (1, 1, 0)^T$ and
%
\begin{eqnarray*}
\bA =  \left[ \begin{array}{rrrr}
     0  &   0  &   1 & -4 \\
     1  &   -2  &   0 & -3 \\
     -1  &   2  &   1 & -1 \\
\end{array} \right].
\end{eqnarray*}
%
Express the solution as a sum of three vectors (including free parameters).

Hint: Feel free to use a computer to help.
%Matlab is permissible (though it is not needed).

\end{enumerate}

%------------------------

\pagebreak
\subsection*{Problem 3 (1.0). Div, grad, curl, and all that}

\citep[][p.~473, Problem~13]{SteinWysession} \\
Consider the vector field
%
\begin{equation}
\bu(x,y,z) = \bxh(3x^2y^2 + z) + \byh(2x^3y+2y) + \bzh(x).
\label{vec}
\end{equation}
%
(Note: Make sure you write down \refEq{vec} correctly; otherwise your answers will all be incorrect!) Calculate the following four expressions:
%
\begin{enumerate}
\item (0.4) $\bdel \cdot \bu$
\item (0.2) $\bdel \times \bu$
\item (0.2) $\nabla^2\bu$
\item (0.2) \ptag\ A scalar field $\phi(x,y,z)$ such that $\bu=\bdel\phi$
\end{enumerate}

%------------------------

%\pagebreak
\subsection*{Problem 4 (1.0). Taylor series}

\begin{enumerate}
\item (0.5) Derive the Taylor series expansion of $f(x) = \ln(x)$ about the point $x = x_0$. List terms up to the third order.

\item (0.5) Let $x_0 = 3.5$. In all of the plots, use $x$ limits from $-1$ to 10 (\verb+xlim([-1 10])+).
%
\begin{enumerate}
\item (0.2) Plot the zeroth-, first-, second-, and third-order {\bf terms} on a single plot.

\item (0.2) Plot the zeroth-, first-, second-, and third-order {\bf Taylor series} on a single plot. Also plot $\ln(x)$.
(The $n$th-order Taylor series is the sum up to the $n$th-order term.) 

\item (0.1) \ptag\ Normalized residuals can be expressed as $\ln(A/B)$, where both $A > 0$ and $B > 0$.
The same can be applied to functions such as $f(x) > 0$ and $g(x) > 0$.
Consider the normalized residuals $\ln[ f(x) / g_k(x) ]$ where $f(x) = \ln(x)$ and $g_k(x)$ is the $k$th-order Taylor series approximation. What is the approximate range of the normalized residual over the range $x = [3,4]$?
\end{enumerate}

\end{enumerate}

%------------------------

%\pagebreak
\subsection*{Problem} \howmuchtime\

%-------------------------------------------------------------
\bibliography{uaf_abbrev,uaf_main,uaf_source,uaf_carletal,uaf_alaska}
%-------------------------------------------------------------

\clearpage\pagebreak
\input{notes_matrix_figs}

%-------------------------------------------------------------
\end{document}
%-------------------------------------------------------------
