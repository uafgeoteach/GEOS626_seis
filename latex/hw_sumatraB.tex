% dvips -t letter hw_sumatraB.dvi -o hw_sumatraB.ps ; ps2pdf hw_sumatraB.ps
\documentclass[11pt,titlepage,fleqn]{article}

\usepackage{amsmath}
\usepackage{amssymb}
\usepackage{latexsym}
\usepackage[round]{natbib}
%\usepackage{epsfig}
\usepackage{graphicx}
\usepackage{bm}

\usepackage{url}
\usepackage{color}
%\usepackage{hyperref}

%--------------------------------------------------------------
%       SPACING COMMANDS (Latex Companion, p. 52)
%--------------------------------------------------------------

\usepackage{setspace}    % double-space or single-space
\usepackage{xspace}

\renewcommand{\baselinestretch}{1.2}

\textwidth 460pt
\textheight 690pt
\oddsidemargin 0pt
\evensidemargin 0pt

% see Latex Companion, p. 85
\voffset     -50pt
\topmargin     0pt
\headsep      20pt
\headheight   15pt
\headheight    0pt
\footskip     30pt
\hoffset       0pt

\input{commands_letters}
\input{commands_uaf}
\input{commands_carl}

\newcommand{\blank}{xxxx}

\newcommand{\cyear}{2026}

% provide space for students to write their solutions
\newcommand{\vertgap}{\vspace{1cm}}

\graphicspath{
  {./figures/}
}

\newcommand{\cltag}{GEOS 626/426: Applied Seismology, Carl Tape}
\newcommand{\ptag}{{\bf \textcolor{magenta}{[GEOS 626]}}}
%\newcommand{\ptag}{}

\bibliographystyle{agufull08}

\newcommand{\howmuchtime}{Approximately how much time {\em outside of class and lab time} did you spend on this problem set? Feel free to suggest improvements here.}

%------------------------------------------------------

% modes
\newcommand{\tnl}[2]{\mbox{$_{#1}\ssT_{#2}$}}
\newcommand{\snl}[2]{\mbox{$_{#1}\ssS_{#2}$}}
\newcommand{\snlm}[3]{\mbox{$_{#1}\ssS_{#2}^{#3}$}}
\newcommand{\Tnl}{\mbox{${}_nT_l$}}   % eigenfun
\newcommand{\Wnl}{\mbox{${}_nW_l$}}   % eigenfun
\newcommand{\Ynl}{\mbox{${}_nY_l$}}   % spherical harmonics
% eigenfrequencies
\newcommand{\omnl}{\mbox{${}_{n\hspace{-0.2 mm}}\omega_l$}}
\newcommand{\fnl}{\mbox{${}_{n\hspace{-0.2 mm}}f_l$}}
\newcommand{\omnlm}[1]{\mbox{${}_{n\hspace{-0.2 mm}}\omega_l^{#1}$}}
\newcommand{\fnlm}[1]{\mbox{${}_{n\hspace{-0.2 mm}}f_l^{#1}$}}

\newcommand{\cutoff}[1]{{#1}_{x\bar{n}}}
\newcommand{\Wcol}{\textcolor{blue}{W}}
\newcommand{\Tcol}{\textcolor{red}{T}}

%\newcommand{\rfind}{{\tt fzero}}
\newcommand{\rfind}{{\tt brentq}}
%\newcommand{\sfind}{{\tt ode45}}
\newcommand{\sfind}{{\tt solve\_ivp}}
\newcommand{\maxstep}{{\tt max\_step}}

%------------------------------------------------------


\newcommand{\tfilemodes}{{\tt hw\_sumatraB\_modes.ipynb}}
\newcommand{\tfilelf}{{\tt hw\_sumatraB\_lf.ipynb}}
\newcommand{\tfilehf}{{\tt hw\_sumatraB\_hf.ipynb}}

%--------------------------------------------------------------
\begin{document}
%-------------------------------------------------------------

\begin{spacing}{1.2} 
\centering
{\large \bf Problem Set 6: Analysis of the 2004 Sumatra-Andaman earthquake \\
Part 2: Analyzing the effects of rupture complexity and Earth heterogeneity [sumatraB]} \\
\cltag\ \\
Assigned: February 25, \cyear\ --- Due: March 18, \cyear\ \\
Last compiled: \today
\end{spacing}

%------------------------

\begin{figure}[h]
\centering
\includegraphics[width=12cm]{sumatra_chen.eps}
\caption[]
{{
Rupture model for the 2004-12-26 \magw{9.2} Sumatra earthquake.
This model, produced by Chen Ji, is a modified version of the one originally presented as Model III in \citet[][Figure~5c]{Ammon2005}.
The color corresponds to the seismological moment associated with each patch (red = large).
Plate boundaries are from \citet{Bird2003}: AU = Australia, IN = India, BU = Burma, SU = Sunda.
}}
\label{fig:sumatra_chen}
\end{figure}

%------------------------

\clearpage\pagebreak
\section*{Overview and Instructions}

\begin{spacing}{1.0}

The purpose of this problem set is to handle a large data set of seismograms and to extract some useful scientific information about the earthquake source (and Earth structure). As you will see, much of the challenge is in representing the seismic waveforms in a sensible manner; this includes judging which seismograms are ``bad,'' that is, not representative of the ground motion. A key part of this representation is determining what bandpass to use for the seismograms for each scientific question.

\begin{itemize}
%\item This problem set uses some scripts in \verb+GEOTOOLS+ for analyzing sets of waveforms. \\
%See \verb+doc_startupB.pdf+ to set up \verb+GEOTOOLS+.

%---------

\item Make sure you have completed:
%
\begin{enumerate}
\item \verb+lab_seismo_rs.pdf+, the lab for getting waveforms and plotting record sections.
Be familiar with what \verb+plotw_rs.py+ is doing.
Be sure you understand what a bandpass filter is.

\item \verb+lab_response.pdf+, the lab on instrument response

\item \verb+lab_sumatra.pdf+, the lab on sumatra waveforms
\end{enumerate}

%---------

\item The main scripts you will use are:
%
\begin{itemize}
\item \tfilemodes\
\item \tfilehf\
\item \tfilelf\
\end{itemize}

%---------

\item Background reading:

\begin{itemize}
\item instrument response and Fourier analysis: \citet[][Ch.~6]{SteinWysession}
\item directivity: \citet[][Section 4.3.2]{SteinWysession} 
\item Sumatra earthquake: \citet{Lay2005,Ammon2005,Park2005,Ni2005,SSteinOkal2007}
\item normal modes: \citet[][Section 2.9]{SteinWysession} and \citet[][Ch.~8]{DT}. See also ``Computational details'' in Section 10.5.1 of DT.
\item PDFs of all referenced Sumatra papers can be found in the class google drive.

%\begin{verbatim}
%/home/admin/databases/SUMATRA/papers/
%/home/admin/databases/SUMATRA/papers/SCIENCE_2005/
%\end{verbatim}

\end{itemize}

%---------

\item To get a better feel for normal mode eigenfunctions, check out this website by Ruedi Widmer--Schnidrig:
%
\begin{verbatim}
http://www.black-forest-observatory.de/Old_Stuttgart_Webpages/modes.html
\end{verbatim}
%
To get a better feel for spherical harmonic functions, try out \verb+ylmplots.ipynb+ and this website:
%
\begin{verbatim}
http://lucien.saviot.free.fr/terre/index.en.html
\end{verbatim}

%---------

\item To subset an obspy Stream by a certain station with an array of stations names \verb+station+, try something like this:
%
\begin{verbatim}
wsub=w.select(station='CAN')
\end{verbatim}

Or to remove traces from stream, try \verb+Stream.remove()+ and \verb+Stream.pop()+:

\begin{verbatim}
wcut=w.remove(station='CAN')
wcut=w.pop(index=2)     
\end{verbatim}
%

\end{itemize}

\end{spacing}

%------------------------

%\pagebreak
\section*{Problem 1 (4.4). Splitting of normal mode frequencies}

The template script is \tfilemodes.
%See note \footnote{In this problem, our spectra are computed from the calibration-applied seismograms; the complete instrument response has not been deconvolved, as we did in the earlier homework. This should not impact our main findings, since we are not computing the {\em relative} amplitudes between modes within a given spectrum.}.

\begin{enumerate}
\item (0.5) Recall the modes spectrum that we computed for station CAN that emulated Figure~1 of \citet{Park2005}. Having done the homework on toroidal modes, you should now have a better understanding of what the modes peaks are.

Run \verb+modes_PREMobs.ipynb+ to see the observed modes that were used in constructing PREM back in 1980 \citep{PREM}. Keep in mind that these observations were identified from dozens of different earthquakes and hundreds of different stations.
The dispersion plot (\refFig{fig:premmodes}) identifies the frequency range 0.2--1.0~mHz that is used in Figure~1 of \citet{Park2005}, which is a spectrum of the {\bf vertical component} of ground motion for one station for one earthquake.
%
\begin{enumerate}
\item (0.1) List the toroidal modes, 0.2--1.0~mHz, that were observed in 1980. \\
List them in increasing frequency.
\item (0.1) List the spheroidal modes, 0.2--1.0~mHz, that were observed in 1980.
\item (0.1) Identify the differences between what was observed in 1980 and what is observed in Figure~1 of \citet{Park2005}.
%\item What toroidal and spheroidal modes are observed (or at least labeled) in Figure~1 of \citet{Park2005} but not observed in 1980?
%\item What toroidal and spheroidal modes observed in 1980 are not present in Figure~1 of \citet{Park2005}?
%\item What two modes are on top of each other?
\item (0.2) Using Ruedi Widmer--Schnidrig's interactive website (above), check out the eigenfunctions for \snl{0}{2}, \snl{1}{2}, and \snl{2}{2}. Describe the differences qualitatively. Explain two reasons why \snl{2}{2} will be difficult to observe.

Reminder of the indexing: \snlm{0}{2}{-1} refers to the $m=-1$ singlet of the mode having degree $l=2$ and overtone number (or radial order) $n=0$.
\end{enumerate}

%---------

\item (0.4) Run \verb+lab_sumatra.ipynb+ and examine the stations that were cut from the analysis. Describe four ``errors'' in these records, and list at least one station associated with each error.

%---------

\item (0.5)
%
\begin{enumerate}
\item (0.4) Modify \verb+ipick+ and plot the spectrum for each station you selected in \verb+lab_sumatra+.
%To save paper, use something like a $5 \times 4$ subplot.

Show a list of station names, along with the distance and azimuth from the Sumatra epicenter to each station.

\item (0.1) Compute the amplitude of the \snl{0}{0} peak for all your stations. \\
List the median value.
\end{enumerate}

%---------

\item (1.0) ``Stacking'' refers to summing similar functions in order to enhance the signal-to-noise ratio. Use the function \verb+w2fstack+ to generate a stack of your spectra.
%
\begin{enumerate}
\item (0.4) Include a plot of the stacked spectrum over two ranges: 0.2--10~mHz and 0.2--1.0~mHz.
\item (0.6) Provide a detailed, qualitative, physical explanation for the occurence of spikes in the 0.2--10~mHz spectrum for any given station. (Recall the modes homework.)
\end{enumerate}
%
% PROBABLY NO NEED TO REMIND EVERYONE ABOUT TIDES HERE, THOUGH IT IS THE LARGEST PEAK IN THE ENTIRE SPECTRUM

%---------

\item (0.5) Several peaks in the Sumatra spectrum are clearly split into multiple peaks known as ``singlets''. For a mode with degree $l$, the theoretical number of singlets is $2l+1$, with the peaks labeled from left to right as $m = -l,\ldots,l$. The central singlet is for $m=0$.
%
\begin{enumerate}
\item (0.2) Use \verb+w2fstack+ to generate a stack of \snl{0}{2}, and include this plot. \\
Hint: Use a new frequency range.
\item (0.1) Label the peaks \snlm{0}{2}{-2}, \ldots, \snlm{0}{2}{2} on the plot.
\item (0.2) 
\begin{itemize}
\item Measure the frequency of the central peak, $_0f_2^0$ ($m=0$), and the spacing between peaks, $\Delta f$. List these values (with units).
\item Assuming a linear model, write an expression for the singlet frequency $_0f_2^m(m)$. List numerical values (and units).
\end{itemize}
\end{enumerate}

%---------

\item (0.5) \ptag\ Consider the variation in \snl{0}{2} based on station latitude.
%
\begin{enumerate}
\item (0.2) What pattern might you expect to see (and why?)

Recall \verb+hw_sumatraA+.
See \citet[][Figure 3]{SSteinOkal2007} for additional background.

\item (0.3) Make a plot with  \snl{0}{2} spectra sorted by station latitude.

Qualitatively, how do the singlet peaks vary as a function of latitude?
%How does the amplitude of the degenerate peak vary with latitude?

\end{enumerate}

%---------

\item (0.5) \ptag\ Now consider the variation in \snl{0}{2} with source-station distance.
%
\begin{enumerate}
\item (0.3) What pattern might you expect to see? \\
Hint: Think about the nodal lines for this mode.

\item (0.2) Make a plot with  \snl{0}{2} spectra sorted by source-station distance. Qualitatively, how does the relative sizes of the singlet peaks vary as a function of source-station distance?
\end{enumerate}

%---------

\item (0.5) Table 5 of \citet{PREM} lists the observed frequencies of spheroidal modes; these are also read in by \verb+modes_PREMobs.ipynb+.
%
\begin{enumerate}
\item (0.2) See if you can identify the peaks for \snl{0}{0}, \snl{1}{0}, and \snl{2}{0} in the stacked Sumatra spectrum. Include your plots.

Hint: You will want to zoom into certain regions of the stacked spectrum.

\item (0.3) What is special about these peaks/modes and why?

See \citet[][p. 106]{SteinWysession} for background.
\end{enumerate}

\end{enumerate}

%------------------------

\pagebreak
\section*{Problem 2 (3.1). Directivity I}

\begin{enumerate}
\item (0.0) Examine \tfilehf\ and make sure you understand how to do certain operations for plotting record sections. It would be helpful to check what each parameter in \verb+plotw_rs.py+ does (type \verb+open plotw_rs+).

\item (1.0) We will repeat the analysis of \citet{Ni2005} but using even more simplifications than they did. {\bf Read \citet{Ni2005} carefully before you begin.} We will assume the following:
%
\begin{itemize}
\item The Earth is flat.
\item The travel time between the fault and any station is encapsulated with the simple velocity $v = 11$~km/s. This is the mean apparent velocity for stations between $30^\circ < \Delta < 85^\circ$ of the source, assuming Jeffreys--Bullen P travel times and arc distances (not distances along the P wave ray path); see \refFig{fig:JB}.
\item All stations are ``far'' from the fault, such that the distance from the station to the starting point is approximately equal to the distance from the station to the stopping point.
\item The fault is positive in length: $L > 0$.
\item The rupture speed is positive and less than the apparent P-wave speed: $0 < v_r < v$.
\end{itemize}
%
The apparent rupture time as measured on a seismogram is given by \citep[][Section 4.3.2]{SteinWysession}
%
\begin{equation}
T_r(\alpha) = L\left(\frac{1}{v_r} - \frac{\cos(\alpha-\alpha_0)}{v}\right)
\label{Tr}
\end{equation}
%
where $v_r$ is the rupture velocity, $v$ is the velocity of the medium (11~km/s), $L$ is the fault length, and $\alpha$ is the azimuth to the station, and $\alpha_0$ is the rupture direction. In answering the following questions, note that {\bf no numbers are needed (only algebra).}
%
\begin{enumerate}
\item (0.1) What is the actual rupture time?
\item (0.1) What $\alpha$ will produce a $T_r$ that is the actual rupture time?
\item (0.1) What $\alpha$ will produce a minimum $T_r(\alpha)$, $T_{\rm min}$?
%What are $\alpha$ values for the minimum and maximum values of $T_r$?
\item (0.1) What $\alpha$ will produce a maximum $T_r(\alpha)$, $T_{\rm max}$?
\item (0.2) The range of $T_r$ is given by $T_{\rm max} - T_{\rm min}$. \\
What is the range, considering variations in $\alpha$ only?
\item (0.1) What is $\overline{T}_r$, the azimuthal average of $T_r$? \\
Hint: Integration is needed.
\item (0.4) Show that, with our assumptions, \refeq{Tr} can be written in terms of only $T_{\rm min}$, $T_{\rm max}$, $\alpha$, and $\alpha_0$.

Hint: \refEq{Tr} is an equation with 6 unknowns: $T_r$, $L$, $v_r$, $v$, $\alpha$, $\alpha_0$. Your equations for $T_{\rm min}$ and $T_{\rm max}$ give you two additional equations with two additional unknowns ($T_{\rm min}$, $T_{\rm max}$). You are asked to write an equation with 5 unknowns (including $T_r$). Therefore you start with a system of 3 equations with 8 unknowns, and you can reduce this to 1 equation with 5 unknowns. This is algebra, so no numbers should appear anywhere.

\end{enumerate}

%--------------------

\item (1.2) 
\begin{enumerate}
\item (1.0) Modify \tfilehf\ to reproduce \citet[][Figure~1d]{Ni2005}, and include a plot of your record section.

Notes:
%
\begin{itemize}
\item See the earlier homework on these waveforms.

\item Note that the epicentral distances must be $30^\circ < \Delta < 85^\circ$, so specify \\
\verb+stasub = [30 85]+ when requesting waveforms.

\item Use at least 10 stations with high-quality seismograms. (Think about what stations would be needed to reproduce the plot in \citet{Ni2005}. Be sure to include CAN from before.)

\item See the template code for how to align seismograms.

\item You do not need to compute the smoothed envelopes---the bandpassed seismograms are enough. (Though computing the envelopes is straightforward; see \verb+hwsol_sumatraA+.)
\end{itemize}

\item (0.2) What are the minimum and maximum time shifts applied to your seismograms in order to align them?

\end{enumerate}

%--------------------

\item (0.4) List three stations with values near $T_{\rm min}$ and three stations with values near $T_{\rm max}$.

List the station name, distance $\Delta$ (degrees), and azimuth $\alpha$ (degrees) for each of station.

%Make a plot of $T_r$ vs $\alpha$ for your data \citep[see Figure~S1 of][]{Ni2005}.
%(This will be very approximate, since it is difficult to measure $T_r$.)

%--------------------

\item (0.4) Using your data, estimate $T_{\rm min}$ and $T_{\rm max}$, then calculate the following quantities:
%
\begin{itemize}
\item rupture direction $\alpha_0$ (this is inferred, not calculated)
\item rupture time $\overline{T}_r$ (show work)
\item rupture length $L$ (show work)
\item rupture velocity $v_r$ (show work)
\end{itemize}

\end{enumerate}

%------------------------

\pagebreak
\section*{Problem 3 (1.0). \ptag\ Directivity II}

\begin{enumerate}
\item (0.4) Modify \tfilelf\ to generate a record section that shows that the directivity can also be inferred from the relative amplitudes of R1 and R2, the ``minor orbit'' and ``major orbit'' Rayleigh wave arrivals\footnote{See Figure 2.7-3 of \citet{SteinWysession}.}. See Figure S1 of \citet{Ammon2005} as a guide.

% XXX PYTHON XXX
Note: You may find the command \verb+extract+ and \verb+max+ useful for quantifying the amplitude ratio, though quantification is not required. If you do quantify the ratios, use the log-scaled quantity $\ln(A_{R1}/A_{R2})$, where $A_{R1}$ is the amplitude of the filtered R1 wave.

\item (0.3) Explain how the R1/R2 ratio can provide information on directivity.

\item (0.3) Is your rupture direction ($\alpha_0$) different for the high-frequency P wave estimate than it is for the long-period Rayleigh wave estimate. Why might this be the case? \\
Hint: Think about \refFig{fig:sumatra_chen}.
\end{enumerate}

%------------------------

%\pagebreak
\section*{Problem 4 (1.5). Miscellaneous}

In this problem, you will find it helpful to use an appropriate input for \verb+stasub+ so that \verb+client.get_waveforms+ will return only a subset of stations/waveforms.

\begin{enumerate}
\item (1.0) Modify \tfilelf\ to show the maximal displacement (not velocity) of some ``near-source'' stations. See Figure~S12 of \citet{Ammon2005} for reference and checking.

%Note: Use \verb+cutoff = [];+ and \verb+iint=0+ when calling \verb+getwaveform+; all filtering and integration can be applied within the plotting script (\verb+plotw_rs.m+).
%
\begin{enumerate}
\item (0.4) Include plots of seismograms (or a record section) for raw data (\verb+iprocess=False+).
\item (0.4) Include plots of seismograms (or a record section) for instrument deconvolved (\verb+iprocess=True+).
\item (0.2) How different are the waveform shapes and amplitudes?
\item (0.0) On the basis of your results, can you infer whether the seismograms in \citet{Ammon2005} were instrument deconvolved or not?
\end{enumerate}
%
Notes:
%
\begin{itemize}
\item Note: Let us assume that \citet{Ammon2005} used the label H1 to denote the E channel (nominally east) and H2 to denote the N channel (nominally north).

\item Make sure you are looking at displacement, not velocity.
%\item Converting to displacement is trivial (see \verb+plotw_rs.py+). If you use \verb+plotw_rs.py+, then you may want to output the modified (\ie filtered) \verb+w+ to analyze.

\end{itemize}

%------------

\item (0.5) \ptag\
%
\begin{enumerate}
\item (0.3) Modify \tfilehf\ to reproduce the azimuthal record section of P waves shown in Figure S11 of \citet{Ammon2005}.
%
\begin{itemize}
\item What bandpass do you think they used?
\item Plot the record section with one column, not two as in \citet{Ammon2005}.
\item Plot (or sort) the seismograms by azimuth.
%\item Order the seismograms starting with the direction of the rupture direction (\verb+azstart+ parameter in \verb+plotw_rs.py+).
\item Hint: The variable \verb+nfac+ will change the amplitudes of the plotted waveforms.
\end{itemize}

\item (0.2) Qualitatively describe the variations in the P waves as a function of azimuth from the rupture direction. (Use the waveforms in Figure S11 of \citet{Ammon2005} if your own record section does not match.)
\end{enumerate}

%------------

\item (0.0) {\bf\em Optional} % embed this text within an enumerated list

Modify {\tt hw\_sumatraB\_P\_hf\_counts.ipynb}
%\tfilephfcounts\
to filter the BHZ time series from \frange{4}{8}. 
I suggest loading the entire data set and browsing all the stations, one seismogram at a time.
%I suggest using the modes-quality stations you selected in Problem 1, but it might be easier to load the entire data set and browse the stations.
Recall what we saw for CAN in the earlier homework.
%
\begin{enumerate}
\item Show a plot and identify other stations (if any) that exhibit high-frequency, post-rupture signals that are similar to those identified at CAN.
\item Is there anything systematic about the other stations? \\
Provide your interpretation of these signals.
\end{enumerate}


\end{enumerate}

%------------------------

%\pagebreak
\subsection*{Problem} \howmuchtime\

%-------------------------------------------------------------
\pagebreak
\bibliography{uaf_abbrev,uaf_carletal,uaf_main,uaf_slab}
%-------------------------------------------------------------

\clearpage\pagebreak
\begin{table}[h]
\caption[Wave parameters and some related equations]
{{
Wave parameters and some related equations.
Note that $\lim_{l\rightarrow \infty } \sqrt{l(l+1)} = l + \hlf$ (see {\tt notes\_wave\_params.pdf}), so in the high-frequency limit, $l + \hlf$ is appropriate.
We have listed ``rad'' for radians, which have no units; however, it is helpful to think of angles for these quantities.
%The equations in the last column incorporate $l$ and $a$ and are approximate for $l \gg 1$ (see \citet{Woodhouse1996}, Eq.~2.110; \citet{DH1986}).
\label{tab:waveparm}
}}
\hspace{-1cm}
\begin{tabular}{||c|c|c|c|c||}
\hline
%
%& & & \\
% ADJUST THE WIDTH OF THE COLUMNS HERE
earth radius & \hspace{10pt} $a$ \hspace{10pt}
& $6.371 \times 10^6$ m & & \\
\cline{1-3}
earth circumference & $2\pi a$ & $4.003 \times 10^7$ m & (with $l$) & (with $l$ and $a$)  \\
%& & & \\
\cline{1-3}
%
%& & & \\
degree	& $l$ & $0,1,2,\ldots$ & & \\
%& & & \\
\hline
%
& & & & \\
period		& $T$
& $T = \frac{1}{f} = \frac{2\pi}{\omega} = \frac{\lambda}{c} = \frac{2\pi}{c k}$ 
& $T = \frac{2\pi}{c\sqrt{l(l+1)}}$ 
& $T = \frac{2\pi a}{c'\sqrt{l(l+1)}}$
\\
& & & [s] & [s] \\ \hline
%
& & & & \\
frequency	& $f$
& $f = \frac{\omega}{2\pi} = \frac{1}{T} = \frac{c}{\lambda} = \frac{ck}{2\pi}$ 
& $f = \frac{c\sqrt{l(l+1)}}{2\pi}$
& $f = \frac{c'\sqrt{l(l+1)}}{2\pi a}$
\\
& & & [1/s] & [1/s] \\ \hline
%
& & & & \\
angular frequency	& $\omega$		
& $\omega = 2\pi f = \frac{2\pi}{T} = ck = \frac{2\pi c}{\lambda}$
& $\omega = c\sqrt{l(l+1)}$
& $\omega = \frac{\;c'\sqrt{l(l+1)}\;}{a}$
\\
& & & [rad/s] & [rad/s] \\ \hline\hline
%
& & & & \\
wavelength	& $\lambda$		
& $\lambda = \frac{c}{f} = c\,T = \frac{2\pi}{k} = \frac{2\pi c}{\omega}$
& $\lambda = \frac{2\pi}{\sqrt{l(l+1)}}$
& $\lambda' = \frac{2\pi a}{\sqrt{l(l+1)}} = \lambda\,a$
\\
& & & [rad] & [m] \\ \hline		
%
& & & & \\
wavenumber		& $k$	
& $k = \frac{\omega}{c} = \frac{2\pi}{\lambda} = \frac{2\pi f}{c} = \frac{2\pi}{Tc}$
& $k = \sqrt{l(l+1)}$
& $k' = \frac{\sqrt{l(l+1)}}{a} = k/a$
\\
& & & [] & [1/m] \\ \hline
%
& & & & \\
phase speed		& $c$			
& $c = \frac{\omega}{k} = \frac{\lambda}{T} = f\lambda = \frac{2\pi f}{k} = \frac{2\pi}{kT}$
& $c = \frac{2\pi}{\;T\sqrt{l(l+1)}\;}$
& $c' = \frac{2\pi a}{\;T\sqrt{l(l+1)}\;} = c\,a$
\\
& & & [rad/s] & [m/s] \\ \hline
& & & & \\
group speed             & $U$
& $U = \frac{d\omega}{d k} = \frac{1}{2\pi}\frac{d f}{d k} = k + \frac{dc}{dk} = c - \frac{dc}{d\lambda}$
& $U$ 
& $U' = U\,a$
\\
& & & [rad/s] & [m/s] \\
\hline
\end{tabular}
\end{table}


\clearpage\pagebreak
\begin{figure}
\centering
\begin{tabular}{cc}
(a) & \includegraphics[width=10cm]{modes_toroidal} \\
(b) & \includegraphics[width=10cm]{modes_spheroidal}
\end{tabular}
\caption[]
{{
Observations of (a) toroidal and (b) spheroidal modes used in making PREM \citep{PREM}.
The horizontal dashed lines mark the frequency range 0.2--1.0~mHz.
These figures are generated from {\tt modes\_PREMobs.ipynb}.
}}
\label{fig:premmodes}
\end{figure}

% see get_JB_Ptime.m
\clearpage\pagebreak
\begin{figure}
\centering
\includegraphics[width=15cm]{sumatra_JBtraveltime.eps}
\caption[]
{{
P wave traveltimes from Jeffreys--Bullen tables.
The calculations are based on a source depth of 10~km.
(a) Apparent P velocity as a function epicentral distances, based on the Jeffreys--Bullen travel time tables. 
The mean velocity over the epicentral distance range $30^\circ < \Delta_{\rm deg} < 85^\circ$ is 10.8~km/s, which would provide a very rough approximation for the P wave velocity for all stations.
(b) Linear fit transformed to traveltime: $P_t = \Delta_{\rm km} \,/\, (0.616 \Delta_{\rm deg} + 7.25)$.
}}
\label{fig:JB}
\end{figure}

%\begin{figure}
%\begin{center}
%\includegraphics[width=15cm]{modes_love_n5_blank.eps}
%\end{center}
%\caption[]
%{{
%Text.
%}}
%\label{fig:love_eigfun_n0}
%\end{figure}


%-------------------------------------------------------------
\end{document}
%-------------------------------------------------------------
